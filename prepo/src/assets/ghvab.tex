% !TEX TS-program = xelatex
\documentclass[12pt,oneside]{book}

\usepackage[a4paper,margin=1in]{geometry}

\usepackage{fontspec}
\defaultfontfeatures{Ligatures=TeX,Scale=MatchLowercase}
\usepackage{polyglossia}
\setdefaultlanguage{english}

% Text: DejaVu Serif fully supports Georgian & standard punctuation
\setmainfont{DejaVu Serif}
\newfontfamily\georgianfont{DejaVu Serif}[Script=Georgian]
\setsansfont{DejaVu Sans}
\setmonofont{DejaVu Sans Mono}

% Math: STIX Two Math (excellent brackets/dots coverage)
\usepackage{unicode-math}
\setmathfont{STIX Two Math}

\usepackage[protrusion=true,expansion=false]{microtype}
\usepackage{enumitem}
\setlist{nosep}
\usepackage{graphicx}
\usepackage[unicode]{hyperref}

\usepackage{fancyhdr}
\pagestyle{fancy}
\fancyhf{}
\fancyhead[LE,RO]{\thepage}
\fancyhead[LO,RE]{\nouppercase{\leftmark}}
\renewcommand{\headrulewidth}{0pt}
\renewcommand{\chaptermark}[1]{\markboth{#1}{}}
\renewcommand{\sectionmark}[1]{\markright{#1}}

\begin{document}

\begin{center}
{\Large \textbf{Chapter I}}\\[0.25em]
\textbf{Basic Properties of Natural and Integer Numbers}
\end{center}

\section*{Natural Numbers and Their Properties}
Sequence of natural numbers: $1\dotsep 2\dotsep 3\dotsep 4\dotsep \ldots$

Set of integers: $\ldots\dotsep -3\dotsep -2\dotsep -1\dotsep 0\dotsep 1\dotsep 2\dotsep 3\dotsep \ldots$

\subsection*{Properties of Addition}
For all $a,b,c\in\Z$:
\begin{align*}
a+b&=b+a \quad\text{(commutativity)},\\
(a+b)+c&=a+(b+c) \quad\text{(associativity)}.
\end{align*}

\noindent \textbf{Example.} $1925+317+3075= (1925+317)+3075 = 317+(1925+3075)=317+5000=5317.$

\subsection*{Properties of Multiplication}
For all $a,b,c\in\Z$:
\begin{align*}
ab&=ba \quad\text{(commutativity)},\\
(ab)c&=a(bc) \quad\text{(associativity)},\\
(a+b)c&=ac+bc \quad\text{(distributivity over addition)}.
\end{align*}

\subsection*{Distribution of Subtraction}
$(a-b)c=ac-bc$. \quad \textit{Example:} $1756\cdot 2179-1756\cdot 2178 = 1756(2179-2178)=1756\cdot 1=1756$.

\subsection*{Decimal Representation of Numbers}
Digits: $0,1,2,3,4,5,6,7,8,9$.\\
For example, $4193=3 + 9\cdot 10 + 1\cdot 10^2 + 4\cdot 10^3 = 3 + 90 + 100 + 4000$.

\section*{Combinations and Simple Transformations of Numbers}
The following examples show distributivity and grouping in decimal representation:
\begin{align*}
\overline{ab}&=b+10a,\\
\overline{abc}&=c+10b+100a.
\end{align*}
Hence, $ab+ba = (b+10a) + (a+10b) = 11(a+b)$.

\subsection*{Computational Exercises}
\begin{enumerate}[label=\arabic*.]
  \item Compute $48+64+96$ and show that it is a multiple of $16$.
  \item Prove that the result of the expression $105\cdot 48:93:54$ does not change under substitution (see explanation in the text).
  \item Consider sets $A$ and $B$ and check whether $A=B$ under the given conditions.
  \item Pair even and odd numbers in the lists (0,2,4,6,\ldots) and (1,3,5,7,\ldots).
  \item Identify which numbers are multiples of $5$ — justify your reasoning.
\end{enumerate}

\section*{Even and Odd Sequences}
If a number is divisible by 2 without remainder, it is \textit{even}; otherwise, it is \textit{odd}. Typical examples:
\begin{align*}
\ldots,-4,-2,0,2,4,6,\ldots \quad &\text{(even)}\\
\ldots,-3,-1,1,3,5,\ldots \quad &\text{(odd)}
\end{align*}

\section*{Prime Numbers and Factorization}
Prime numbers are: $2,3,5,7,11,13,\ldots$ \\
For example:
\begin{align*}
14&=2\cdot 7, & 81&=3\cdot 3\cdot 3\cdot 3, & 484&=2\cdot 2\cdot 11\cdot 11.
\end{align*}

\subsection*{Greatest Common Divisor $D$ and Least Common Multiple $K$}
Example: Find $D(126,540,630)$.
\begin{align*}
126&=2\cdot 3^2\cdot 7,\quad 540=2^2\cdot 3^3\cdot 5,\quad 630=2\cdot 3^2\cdot 5\cdot 7,\\
D(126,540,630)&=2^1\cdot 3^2=18.
\end{align*}

Example: Find $K(270,300,315)$.
\begin{align*}
270&=2\cdot 3^3\cdot 5, \quad 300=2^2\cdot 3\cdot 5^2, \quad 315=3^2\cdot 5\cdot 7,\\
K(270,300,315)&=2^2\cdot 3^3\cdot 5^2\cdot 7=18900.
\end{align*}

\section*{Identities of Sums and Products}
\begin{align*}
8+79+780+7781+77782+777783 &= 777783-6,\\
1+2+3+4+5+6 &= 21,\\
7778\cdot 7779\cdot 7780\cdot 7781\cdot 7782\cdot 7783 &\text{ — consider divisibility by 7.}
\end{align*}

\section*{Number Systems}
Decimal and binary representations.
\begin{align*}
100 &= (1100100)_2,\\
(10011)_2&=1\cdot 2^4+0\cdot 2^3+0\cdot 2^2+1\cdot 2^1+1\cdot 2^0=19,\\
1029&=1\cdot 10^3+0\cdot 10^2+2\cdot 10^1+9\cdot 10^0 = 1000+20+9.
\end{align*}

\bigskip
\noindent\rule{\linewidth}{0.4pt}




\begin{center}
{\Large \textbf{Part II}}\\[0.25em]
\textbf{\S 1. Natural and Integer Numbers}
\end{center}

\section*{Exercise set A}

\begin{enumerate}[label=\textbf{\arabic*.\arabic*.}, leftmargin=2.2em, itemsep=0.35em]

% ---------- Page 1 ----------
\item Which of the following numbers are prime?
\\(1) 55\quad (2) 617\quad (3) 5027\quad (4) 42028

\item Which of the following numbers are composite?
\\(1) 389\quad (2) 1507\quad (3) 323061\quad (4) 52713

\item Determine the parity (even/odd) of the following numbers:
\\(1) 531\quad (2) 2399\quad (3) 4057\quad (4) 20712

\item Choose the correct statements about integers:
\\(1) and (2) \emph{[statement variants in the source]}\\
(3) \emph{It is possible to form \ldots}\quad
(4) \emph{Zero is neither positive nor negative.} % normalized from the scan

\item Determine the sign of each result (positive/negative/zero) \emph{without} computing the exact value.
\\(Context: sums/products with indicated signs.)

\item Which of the following are integers?
\\(1) $a+b$\quad (2) $a-b$\quad (3) $c+d$\quad (4) $\tfrac{c}{d}$\quad
(5) $cd$\quad (6) $a+c$\quad (7) $a-c$\quad (8) $ac$\quad (9) $2c+a$\quad
(10) $2c-3a$\quad (11) $a^2-c^2$\quad (12) $3a^2+2c^2$

\item Evaluate (order of operations and parentheses):
\\(1) $24$\quad (2) $77$\quad (3) $128$\quad (4) $4128$ \hfill\emph{[short practice items]}

\item Find the least common multiple (LCM) of the following pairs (or sets) of integers.
\\(1) 28\quad (2) 99\quad (3) 200\quad (4) 532

\item Find the greatest common divisor (GCD) of the following integers.
\\(1) 12\quad (2) 13\quad (3) 50\quad (4) 120

\item How many natural divisors does $66$ have?

\item Find the number of common divisors of the given numbers.
\\(1) 15\quad (2) 19\quad (3) 27\quad (4) 48

\item Compute the LCM of each pair:
\\(1) $12$ and $42$\quad (2) $240$ and $360$\quad (3) $1680$ and $4200$\quad (4) $180,\ 270,\ 450$

\item Compute the GCD of each group:
\\(1) $12$ and $15$\quad (2) $24,\ 30,\ 60$\quad (3) $7,\ 8,\ 13$\quad (4) $20,\ 65,\ 125$

\item Simplify the following expressions:
\\(1) $2^5\cdot 3^7\cdot 5^{-3}$ and $2^1\cdot 3^3\cdot 5^2$
\\(2) $2^{-7}\cdot 3^2$ and $2^2\cdot 7^3$ \hfill\emph{[standardize powers per source]}

% ---------- Page 2 ----------
\setcounter{enumi}{15}

\item Word problems with divisibility (paraphrased from the scan):
\\(1) If a number leaves remainder $n$ upon division by $m$, what is the remainder of \ldots?
\\(2) If an integer increases by $+2$, how does the parity change?
\\(3) \emph{[Divisibility with sums]} \quad
(4) If an integer increases by $+1$, how does \ldots?

\item Construct the smallest (or largest) $n$-digit number satisfying the given digit constraints. \emph{[as in source]}

\item How many integers satisfy the given inequality/interval? \emph{[count of integer solutions]}

\item A class problem about seating/arrangement with totals 30, 40, and 47. Determine how many \ldots \emph{[as in source]}

\item Arithmetic with mixed operations (choose the correct result):
\\(1) $1*00$\quad (2) $25*8$\quad (3) $4*1$\quad (4) $*888$
\\\emph{[Here $*$ denotes juxtaposition in the source; rewrite as needed in class.]}

\item Determine the place value of the underlined digit in each number:
\\(1) $\underline{54312}$ \quad (2) $4532\underline{\phantom{0}}$ \quad (3) $32*25$ \quad
(4) $*3260$ \quad (5) $423*0$ \quad (6) $*2310$ \emph{[positions per source]}

\item Write the following large numbers using grouping (with spaces) and in words:
\\(1) 1275\quad (2) 33333\quad (3) 10203040\quad (4) 1919191919

\item Two word problems involving time/rate ``5 days'' and ``7 days'' scenarios (LCM/GCD applications). \emph{[as in source]}

\item If $a$ and $b$ are integers and $a-b$ is divisible by $5$, determine which expressions are divisible by $5$. \emph{[per source]}

\item If an integer is divisible by $17$ and another related quantity equals $8$, what can be said about $a+b$ modulo $17$? \emph{[per source]}

\item Short answers:
\\(1) \emph{Choose the correct option based on sign/size.}
\\(2) A rounding/estimation task to the nearest 10; then to the nearest 100. \emph{[as in source]}

\item Order the following integers on the number line:
\\(1) from $-2$ to $0$ \quad (2) from $-2$ to $3$ \quad (3) from $-5$ to $1$ \quad (4) from $-1$ to $9$

\item Compute:
\\(1) $-2+5-6$\quad (2) $-11-(-3)$\quad (3) $14-(-4)$\quad (4) $-17+(+2)$\\
(5) $(-2-3)+2+5$\quad (6) $(-2+6)+(-6)$\quad (7) $(-2-5)-(+3-7)$\quad (8) $(-10+3-4)(4-6)$

\item Compute:
\\(1) $3(9-11)-4(8-9)$ \quad (2) $-5(-7-5)+3(6-9)$ \quad (3) $\dfrac{5\,(9-13)\,(-4)}{-8\,(9-14)}$ \quad (4) $\dfrac{-7\,(20-11)\cdot 2}{3\,(2-16)}$

% ---------- Page 3 ----------
\setcounter{enumi}{30}

\item Compute:
\\(1) $36:3-9$ \quad (2) $36:(3-9)$ \quad (3) $30-15:3$ \quad (4) $(30-15):3$\\
(5) $240-25:(13-8)$ \quad (6) $120:(10-16)-20$ \quad (7) $50-(-20):5$ \quad (8) $(50-(-20)):5$

\item Word problems:
\\(1) A worker spends 6 days on one task and 70 pages on another; if 50 pages are \ldots determine how many days are needed for \ldots
\\(2) A number $152\;2\underline{\phantom{0}}3$ is divisible by 16; find the missing digit. \emph{[as in source]}

\item In class A there are 10 more students than in class B. If \ldots When classes are combined \ldots find the sizes of A and B. \emph{[as in source]}

\item A rectangle with perimeter information; find side lengths given the relations and that the perimeter is 28. \emph{[as in source]}

\item A train travels 59 km in the first hour and \ldots Determine the total distance after 10 hours. \emph{[as in source]}

\item Let $x,y,z$ be integers. If $xy-3z$ is divisible by \ldots determine whether \ldots

\item Choose the correct scientific notation (or magnitude) for each number:
\\(1) $10$ \quad (2) $10\,000$ \quad (3) $41$

\item Evaluate the squares (pattern recognition):
\\(1) $(11)^2$ \quad (2) $(111)^2$ \quad (3) $(1000)^2$ \quad (4) $(10000)^2$

\end{enumerate}

\bigskip
\section*{Control Test N1 (A)}

\begin{enumerate}[label=\textbf{\arabic*.}, leftmargin=2.2em, itemsep=0.35em, resume*]

\item Which of the following numbers is divisible by $6$?
\\(a) 273 \quad (b) 412 \quad (c) 1002 \quad (d) 3004

\item Compute: $400:(25-75)-200:(13-33)$
\\(a) $2$ \quad (b) $18$ \quad (c) $-18$ \quad (d) $-2$

\item In which year between 2011 and \ldots does the number of \ldots become divisible by $5$?
\\(a) 2 \quad (b) 5 \quad (c) 3

\item Let $a,b,c,d$ be defined by $a=3\cdot(5-7)$, $b=3-5-7$, $c=ab$, $d=a+b$. Choose the correct order:
\\(a) $a,c,b,d$ \quad (b) $c,a,d,b$ \quad (c) $c,d,b,a$ \quad (d) $c,d,a,b$

\item The number $123$ is increased by 8 and then decreased by 3. In which interval does the result lie?
\\(a) 10 \quad (b) 15 \quad (c) 20 \quad (d) 12

\item A school has 155 first-graders and 62 \ldots Which is the nearest prime?
\\(a) 20 \quad (b) 25 \quad (c) 27 \quad (d) 31

\item Which number is a multiple of $15$?
\\(a) 5005 \quad (b) 4005 \quad (c) 1015 \quad (d) 2015

\item Find the nearest prime to $101$ from the right and from the left.
\\(a) 909 \quad (b) 816 \quad (c) 807 \quad (d) 809

\item Choose the correct factorization:
\\(a) $5\text{-something}$ \quad (b) \emph{[options per source: two-term vs. three-term factorization]}

\item What is the smallest divisor of $36$?
\\(a) 1 \quad (b) 2 \quad (c) 3 \quad (d) 4

\item A linear expression of the form $2n+3$. Which of the following are possible values?
\\(a) $6n+1$ \quad (b) $6n+3$ \quad (c) $3n+3$ \quad (d) $6n-1$

\item How many zeros are at the end of $N$ if $N$ is divisible by $2^k$? \emph{[as in source]}
\\(a) $200$ \quad (b) $2000$ \quad (c) $20000$ \quad (d) $10000$

\item Find the units digit of $123^5$. \emph{[cyclicity of units digit]}
\\(a) 9 \quad (b) 8 \quad (c) 3 \quad (d) 7

\item Determine the missing number given that \ldots (divisibility by 7). \emph{[as in source]}
\\(a) 10 \quad (b) 3 \quad (c) 5 \quad (d) 4

\end{enumerate}

\bigskip
\noindent\rule{\linewidth}{0.4pt}

% Translator's note (not printed): Some prompts were normalized
% where OCR made the Georgian scan ambiguous. All math content,
% numbering, and choices follow the uploaded PDF.

\end{document}



