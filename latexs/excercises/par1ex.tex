\documentclass{article}
\usepackage[utf8]{inputenc}
\usepackage[T1]{fontenc}
\usepackage[georgian]{babel}
\usepackage{amsmath, amssymb}
\usepackage{geometry}
\geometry{a4paper, margin=1in}

\usepackage{fontspec}
\setmainfont{DejaVu Sans}
\usepackage[margin=2cm]{geometry}

\usepackage{amssymb}
\usepackage{multicol}
\usepackage[inline]{enumitem} % inline lists enabled
\usepackage{tikz}

% --- Make all math the same size as normal text (12pt) ---
\AtBeginDocument{%
  \DeclareMathSizes{12}{12}{12}{12}%
}
\everydisplay{\textstyle}


\begin{document}

\section*{ნაწილი II}

\subsection*{§ 1. ნატურალური და მთელი რიცხვები}

\begin{enumerate}
\item[1.1.] სულ რამდენი ათეულია შემდეგ რიცხვებში?
\begin{enumerate}
\item 55;
\item 617;
\item 5027;
\item 42028.
\end{enumerate}

\item[1.2.] სულ რამდენი ასეულია შემდეგ რიცხვებში?
\begin{enumerate}
\item 389;
\item 1507;
\item 323061;
\item 52713.
\end{enumerate}

\item[1.3.] რიცხვები ჩაწერეთ თანრიგების მიხედვით:
\begin{enumerate}
\item 531;
\item 2399;
\item 4057;
\item 20712.
\end{enumerate}

\item[1.4.] ჩაწერეთ შემდეგი რიცხვები:
\begin{enumerate}
\item სამი ათას ოცდახუთი;
\item ათი ათას ერთი;
\item ოთხმოცდაცხრამეტი ათას ოთხმოცდაცხრამეტი;
\item სამი მილიონ ოთხმოცდათოთხმეტი ათას ოთხი.
\end{enumerate}

\item[1.5.] ჩაწერეთ ყველა ორნიშნა რიცხვი, რომელიც მიიღება მოცემული ციფრებით. ყოველი ციფრი გამოიყენეთ მხოლოდ ერთხელ:
\begin{enumerate}
\item 2, 5 და 7;
\item 3, 0 და 1.
\end{enumerate}

\item[1.6.]
\begin{enumerate}
\item რამდენი ორნიშნა რიცხვი არსებობს?
\item რამდენი კენტი ორნიშნა რიცხვი არსებობს?
\item რამდენი სამნიშნა რიცხვი არსებობს?
\item რამდენი ლუწი სამნიშნა რიცხვი არსებობს?
\end{enumerate}

\item[1.7.] $a$ და $b$ ლუწი რიცხვებია, $c$ და $d$ კი კენტი. ლუწია თუ კენტი:
\begin{enumerate}
\item $a+b$;
\item $a-b$;
\item $c+d$;
\item $c-d$;
\item $cd$;
\item $a+c$;
\item $a-c$;
\item $ac$;
\item $2c+a$;
\item $2c-3a$;
\item $a^{2}-c^{2}$;
\item $3a^{2}+2c^{2}$.
\end{enumerate}

\item[1.8.] დაშალეთ მარტივ მამრავლებად:
\begin{enumerate}
\item 24;
\item 77;
\item 128;
\item 4128.
\end{enumerate}

\item[1.9.] იპოვეთ შემდეგი რიცხვების უდიდესი მარტივი გამყოფი:
\begin{enumerate}
\item 28;
\item 99;
\item 200;
\item 532.
\end{enumerate}

\item[1.10.] იპოვეთ შემდეგი რიცხვების მარტივი გამყოფების რაოდენობა:
\begin{enumerate}
\item 12;
\item 13;
\item 50;
\item 120.
\end{enumerate}

\item[1.11.] რას უდრის 66-ის მარტივ გამყოფთა ჯამი?

\item[1.12.] იპოვეთ შემდეგი რიცხვების ნატურალური გამყოფების რაოდენობა:
\begin{enumerate}
\item 15;
\item 19;
\item 27;
\item 48.
\end{enumerate}

\item[1.13.] მამრავლებად დაშლის ხერხით იპოვეთ შემდეგი რიცხვების უდიდესი საერთო გამყოფი:
\begin{enumerate}
\item 12 და 42;
\item 240 და 360;
\item 1680 და 4200;
\item 180, 270 და 450.
\end{enumerate}

\item[1.14.] მამრავლებად დაშლის ხერხით იპოვეთ შემდეგი რიცხვების უმცირესი საერთო ჯერადი:
\begin{enumerate}
\item 12 და 15;
\item 24, 30 და 60;
\item 7, 8 და 13;
\item 20, 65 და 125.
\end{enumerate}

\item[1.15.] იპოვეთ რიცხვების უდიდესი საერთო გამყოფი:
\begin{enumerate}
\item $2^{5} \cdot 3^{7} \cdot 5$ და $3^{3} \cdot 5^{2}$;
\item $2 \cdot 7^{2} \cdot 3^{2}$ და $2^{2} \cdot 7^{3}$.
\end{enumerate}

\item[1.16.]
\begin{enumerate}
\item სამ მომდევნო ნატურალურ რიცხს შორის უმცირესია $n$, რას უდრის უდიდესი?
\item ხუთ მომდევნო ნატურალურ რიცხს შორის უდიდესია $m+2$, რას უდრის უმცირესი?
\item ჩაწერეთ სამი მომდევნო ნატურალური რიცხვის ნამრავლი, თუ მათგან უმცირესია $n$.
\item ჩაწერეთ სამი მომდევნო ნატურალური რიცხვის ნამრავლი, თუ მათგან უდიდესია $m+1$.
\end{enumerate}

\item[1.17.] რა ციფრით შეიძლება დამთავრდეს ნატურალური რიცხვის კვადრატი?

\item[1.18.] 9-ის ჯერადი რამდენი ორნიშნა რიცხვი არსებობს?

\item[1.19.] გაკვეთილის ბოლოს მოსწავლეებმა ჩააბარეს საკონტროლო წერის და სავარჯიშო რვეულები, სულ 47. ყველა მოსწავლემ ჩააბარა თუ არა ორი რვეული?

\item[1.20.] ბოლო გაჩერებიდან ორი მარშრუტით გამოდის ავტობუსები. პირველი ბრუნდება უკან ყოველ 30 წუთში, მეორე კი ყოველ 40 წუთში. რა უმცირეს დროში აღმოჩნდებიან ისინი ერთდროულად ბოლო გაჩერებაზე?

\item[1.21.] ვარსკვლავის ნაცვლად ჩაწერეთ ისეთი ციფრი, რომ მიიღოთ რიცხვი, რომელიც იყოფა 9-ზე:
\begin{enumerate}
\item 1*00;
\item 25*8;
\item 4*1;
\item *888.
\end{enumerate}

\item[1.22.] რა ციფრი შეიძლება ეწეროს ვარსკვლავის ნაცვლად, თუ ცნობილია, რომ:
\begin{enumerate}
\item *54312 იყოფა 3-ზე;
\item 4532* იყოფა 5-ზე;
\item 32*25 იყოფა 5-ზე;
\item *3260 იყოფა 5-ზე;
\item 423*0 იყოფა 10-ზე;
\item *2310 იყოფა 10-ზე.
\end{enumerate}

\item[1.23.] რა უმცირესი რიცხვი უნდა მივუმატოთ ჩამოთვლილ ნატურალურ რიცხვებს, რომ მივიღოთ 9-ის ჯერადი რიცხვი:
\begin{enumerate}
\item 1275;
\item 33333;
\item 10203040;
\item 1919191919.
\end{enumerate}

\item[1.24.] იპოვეთ 5 რიცხვი, რომლებიც:
\begin{enumerate}
\item 5-ზე გაყოფისას ნაშთში იძლევიან 3-ს და აჩვენეთ, რომ ნებისმიერი ორი მათგანის სხვაობა იყოფა 5-ზე;
\item 7-ზე გაყოფისას ნაშთში იძლევიან 5-ს და აჩვენეთ, რომ ნებისმიერი ორი მათგანის სხვაობა იყოფა 7-ზე.
\end{enumerate}

\item[1.25.] $a$ და $b$ მთელი რიცხვების 5-ზე გაყოფისას ერთსა და იმავე ნაშთს იღებენ. იყოფა თუ არა $a-b$ სხვაობა 5-ზე? (პასუხი დაასაბუთეთ).

\item[1.26.] $a$ რიცხვის 17-ზე გაყოფისას მიიღება ნაშთი 9, ხოლო $b$ რიცხვის 17-ზე გაყოფისას კი – 8. რა ნაშთი მიიღება $a+b$-ს 17-ზე გაყოფისას?

\item[1.27.]
\begin{enumerate}
\item რას უდრის სხვაობა უმცირეს ნატურალურ ხუთნიშნა რიცხვსა და უდიდეს ნატურალურ ოთხნიშნა რიცხვს შორის?
\item რამდენით მეტია უდიდესი სამნიშნა რიცხვი, რომლის ციფრთა ჯამი 10-ის ტოლია, უმცირეს სამნიშნა რიცხვზე, რომლის ციფრთა ჯამიც ასევე 10-ის ტოლია?
\end{enumerate}

\item[1.28.] რომელი მთელი რიცხვებია მოთავსებული:
\begin{enumerate}
\item –2-სა და 0-ს შორის;
\item –2-სა და 3-ს შორის;
\item –5-სა და 1-ს შორის;
\item –1-სა და 0-ს შორის.
\end{enumerate}

\item[1.29.] გამოთვალეთ:
\begin{enumerate}
\item –2+5–6;
\item –11–(–3);
\item 14–(–4);
\item –17+(–2);
\item (–2–3)–(–2+5);
\item (–2+6)–(–6);
\item (–2–5)–(–3+7);
\item (–10+3–4)(4–6);
\end{enumerate}

\item[1.30.] გამოთვალეთ:
\begin{enumerate}
\item 3-(9–11)–4-(8–9);
\item –5-(7–5)+3-(6–9);
\item $\frac{5 \cdot (9-13) \cdot (-4)}{-8 \cdot (9-14)}$;
\item $\frac{-7 \cdot (20-11) \cdot 2}{3 \cdot (2-16)}$.
\end{enumerate}

\item[1.31.] გამოთვალეთ:
\begin{enumerate}
\item 36:3-9;
\item 36:(3-9);
\item 30-15:3;
\item (30-15):3;
\item 240-25:(13-8);
\item 120:(10-16)-20;
\item 50-(-20):5;
\item (50-(-20)):5.
\end{enumerate}

\item[1.32.]
\begin{enumerate}
\item კონფერენცია 6 დღე გრძელდებოდა და ყოველდღე იხარჯებოდა 70 პაკეტი ჩაი. ჩაი იყიდება 50 პაკეტიანი ყუთებით. რამდენი ყუთი ჩაის ყიდვა დასჭირდათ ორგანიზატორებს?
\item საზაფხულო ბანაკში 152 ბავშვი და 16 აღმზრდელია. რამდენი ავტობუსი იქნება საჭირო მათი გადაყვანად, თუ ერთ ავტობუსში ეტევა არა უმეტეს 44 მგზავრისა?
\end{enumerate}

\item[1.33.] Α კლასში 10-ით მეტი მოსწავლეა, ვიდრე Β კლასში. რამდენით მეტი იქნება Α კლასში მოსწავლეთა რაოდენობა Β კლასში მოსწავლეთა რაოდენობაზე, თუ Α კლასიდან Β კლასში 3 მოსწავლე გადავა?

\item[1.34.] მას შემდეგ, რაც კლასში 9 მოსწავლე მოვიდა და კლასიდან 6 მოსწავლე წავიდა, კლასში 28 მოსწავლე აღმოჩნდა. რამდენი მოსწავლე იყო კლასში თავდაპირველად?

\item[1.35.] ერთ საწყობში 50 ტონით მეტი ნავთობია, ვიდრე მეორეში. რამდენი ტონა ნავთობი უნდა გადავიტანოთ პირველიდან მეორე საწყობში, რომ პირველ საწყობში იყოს 10 ტონით მეტი ნავთობი, ვიდრე მეორეში?

\item[1.36.] თუ $x$, $y$, $z$ რიცხვები უარყოფითია, მაშინ $xy - 3z$ გამოსახულება უარყოფითია თუ დადებითი?

\item[1.37.] რიცხვები ათობითი სისტემიდან გადაიყვანეთ ორობით სისტემაში:
\begin{enumerate}
\item 10;
\item 10000;
\item 41.
\end{enumerate}

\item[1.38.] რიცხვები ორობითი სისტემიდან გადაიყვანეთ ათობით სისტემაში:
\begin{enumerate}
\item $(11)_2$;
\item $(1111)_2$;
\item $(1000)_2$;
\item $(10000)_2$.
\end{enumerate}

\end{enumerate}

\end{document}