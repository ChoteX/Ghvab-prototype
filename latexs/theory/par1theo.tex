\documentclass[12pt,a4paper]{article}
\usepackage{fontspec}
\setmainfont{DejaVu Sans}
\usepackage[margin=2cm]{geometry}
\usepackage{amsmath}
\usepackage{amssymb}
\usepackage{multicol}
\usepackage[inline]{enumitem} % inline lists enabled
\usepackage{tikz}

% --- Make all math the same size as normal text (12pt) ---
\AtBeginDocument{%
  \DeclareMathSizes{12}{12}{12}{12}%
}
\everydisplay{\textstyle}

\begin{document}

\begin{titlepage}
    \centering
    \vspace*{2cm}
    {\Large ბ. ღვაბერიძე, თ. დვალიშვილი, ა. მოსიძე, ე. გელაშვილი, თ. სირბილაძე\par}
    \vspace{2cm}
    {\Huge \textbf{მათემატიკა}\par}
    \vspace{0.5cm}
    {\Large \textbf{სასწავლებო და ეროვნული გამოცდებისთვის}\par}
    \vspace{0.5cm}
    {\large \textbf{მეხუთე გამოცემა}\par}
    \vspace{1cm}
    {\large \begin{tabular}{l}
        ✓ თეორია \\
        ✓ ამოცანათა კრებული \\
        ✓ პასუხები \\
        ✓ ტესტები
    \end{tabular}\par}
    \vfill
    {\large 2011\par}
\end{titlepage}

\begin{titlepage}
    \centering
    \vspace*{2cm}
    {\small ბესარიონ ღვაბერიძე, თენგიზ დვალიშვილი, ალექსანდრე მოსიძე, \\
    კობა გელაშვილი, თინა სირბილაძე\par}
    \vspace{3cm}
    {\Huge \textbf{მათემატიკა}\par}
    \vspace{0.5cm}
    {\Large ალგებრა და ანალიზის საფუძვლები\par}
    \vspace{0.5cm}
    {\large მეხუთე გამოცემა\par}
    \vfill
    {\large \textbf{პროექტის} \par}
    \vspace{0.5cm}
    {\large \textbf{პროექტი თბილისი 2011}\par}
\end{titlepage}

\vspace{1cm}

სასწავლო-მეთოდიკური განკუთვნილია როგორც პედაგოგებისთვის, ასევე დამამთავრებელი კლასის მოსწავლეებისთვის. წარმოდგენილია პროგრამული მასალა XI და XII კლასებისათვის, მოცემულია თეორიის შინაარსობრივი და ფორმალური მხარეები, რომლებიც მოსწავლეს საშუალებას მისცემს უკეთ გაიაზროს სწავლებული მასალა და წარმატებით მოემზადოს გამოცდებისთვის. წიგნში მოცემულია ამოცანების კრებული, პასუხები და ტესტები, რომლებიც ხელს შეუწყობს მოსწავლის ცოდნის გაღრმავებას და მიღებული ცოდნის შემოწმებას.

\vspace{0.5cm}

\textbf{რედაქტორი} -- განათლების სამინისტროს მეთოდიკური კომიტეტის წევრი \\
გ. კაკაბაძე

\vspace{0.5cm}

\textbf{რეცენზენტები:}
\begin{enumerate}
    \item თბილისის საჯარო სკოლების მასწავლებლის უმაღლესი კატეგორიის, ფიზიკა-მათემატიკის მეთოდიკური კომიტეტის წევრი \\
    ს. ხაჩიძე
    \item ქ. თბილისის 199-ე საჯარო სკოლის პედაგოგი \\
    გ. მანგოშვილი
\end{enumerate}

\vspace{0.5cm}

ყველა უფლება დაცულია. წიგნის არც ერთი ნაწილი არ შეიძლება იქნეს გადმოწერილი, ასახული ან გავრცელებული ავტორის წერილობითი ნებართვის გარეშე.

\vspace{0.5cm}

© ბ. ღვაბერიძე, 2011 \\
© გამომცემლობა \textquotedblleft პროექტი\textquotedblright, თბილისი, 2011 \\
ISBN 978-9941-0-3645-3

\newpage
\section*{წინასიტყვაობა}

სასწავლო პროცესში განკუთვნილი როგორც აბიტურიენტებისთვის, ასევე დამამთავრებელი კლასის მოსწავლეებისთვის.

მშრალი ფორმულის მიცემა ხშირად რთულად ქცევს საკითხის გაგებას და "მექანიკურ" დამახსოვრებას. ჩვენ წიგნში "S" ნიშნის გამოყენებით და ტექსტში "მუქი" ფონით, ჩართულია დამამახსოვრებელი კლასიკური ფორმულები, რაც მკითხველს საშუალებას აძლევს სწრაფად იპოვოს საჭირო ინფორმაცია. "S" ნიშანი აღნიშნავს ფორმულას, რომელიც აუცილებელია, ხოლო "მუქი" ფონით გამოყოფილია ფორმულები, რომლებიც ხშირად გვხვდება პრაქტიკაში.

მოსწავლეებისთვის სასურველია ნასწავლი თემის თეორიული მასალის შემდეგ გადაჭრან შესაბამისი სავარჯიშოები, რაც ხელს უწყობს ცოდნის გამყარებას. წიგნში მოცემულია როგორც თეორიული მასალა, ასევე მაგალითები და სავარჯიშოები, რომლებიც თანდათანობით რთულდება.

მიუხედავად იმისა, რომ წიგნი განკუთვნილია აბიტურიენტებისთვის, იგი შეიძლება გამოყენებულ იქნას როგორც დამხმარე მასალა სკოლის მასწავლებლებისთვისაც.

წიგნის ბოლოში მოცემულია პასუხები ყველა სავარჯიშოზე.

ავტორის მიზანია, რომ ეს წიგნი დაეხმაროს მოსწავლეებს და აბიტურიენტებს მათემატიკის უკეთ გაგებაში.

სასწავლო პროცესში ჩართულია როგორც თეორიული მასალა, ასევე პრაქტიკული მაგალითები, რაც სასწავლო პროცესს უფრო საინტერესო და სასარგებლოს ხდის.

განსაკუთრებული მადლობა მინდა გადავუხადო ჩემს კოლეგებს, რომლებიც დამეხმარნენ წიგნის შექმნაში, ასევე ჩემს მეგობრებს, რომლებმაც თავიანთი რჩევებით და რეკომენდაციებით ხელი შეუწყვეს ამ წიგნის შექმნას.

\section*{სარჩევი}

\subsection*{ნაწილი I}
\begin{tabular}{ll}
§1. & ნატურალური და მთელი რიცხვები \dotfill 7 \\
§2. & რიცხვთა ნაკრები \dotfill 14 \\
§3. & სარიცხვო რგოლები: რიცხვთა შულები, რგოლის მოდული \dotfill 21 \\
§4. & რიცხვების გამოკლება \dotfill 27 \\
§5. & რიცხვების გამრავლება \dotfill 34 \\
§6. & პროპორცია. პროპორციის თვისებები \dotfill 38 \\
§7. & რიცხვთა ნაკრების სიდიდე და სიდიდეთა შედარება. წილები, წილადის კრებული \dotfill 41 \\
§8. & წილადის გამოკლება. წილადის გამრავლება \dotfill 47 \\
§9. & წილადის გაყოფა. წილადის გაძლიერება და დაქვეითება. წილადის გაერთიანება და დაშლა \dotfill 52 \\
§10. & წილადის გამრავლება. წილადის გაძლიერება და დაქვეითება \dotfill 57 \\
§11. & წილადის გაყოფა. წილადის გაძლიერება და დაქვეითება. წილადის გაერთიანება და დაშლა \dotfill 69 \\
§12. & $y = \frac{k}{x}$, $y = \sqrt{x}$ და $y = x^2$ ფუნქციები \dotfill 74 \\
§13. & ტოლობის მახასიათებლები \dotfill 75 \\
§14. & ტოლობის გადაჭრა \dotfill 78 \\
§15. & არათოლობა \dotfill 84 \\
§16. & არათოლობა და ლოგარითმული ფუნქცია. არათოლობის გადაჭრა \dotfill 93 \\
§17. & არათოლობა და ლოგარითმული ფუნქცია. არათოლობის გადაჭრა \dotfill 99 \\
§18. & არათოლობა და ლოგარითმული ფუნქცია \dotfill 107 \\
\end{tabular}

\subsection*{ნაწილი II}
\begin{tabular}{ll}
§1. & ნატურალური და მთელი რიცხვები \dotfill 119 \\
§2. & რიცხვთა ნაკრები \dotfill 130 \\
§3. & სარიცხვო რგოლები: რიცხვთა შულები, რგოლის მოდული, მოდულის სიდიდე \dotfill 141 \\
§4. & რიცხვების გამოკლება. რიცხვების გამრავლება \dotfill 148 \\
§5. & პროპორცია. პროპორციის თვისებები \dotfill 160 \\
§6. & რიცხვთა ნაკრების სიდიდე და სიდიდეთა შედარება. წილები, წილადის კრებული \dotfill 172 \\
\multicolumn{2}{l}{ტესტი გამოცდისათვის №1} \\
§7. & წილადის გამოკლება. წილადის გამრავლება \dotfill 185 \\
§8. & წილადის გაყოფა. წილადის გაძლიერება და დაქვეითება. წილადის გაერთიანება და დაშლა \dotfill 205 \\
§9. & წილადის გამრავლება. წილადის გაძლიერება და დაქვეითება \dotfill 222 \\
§10. & წილადის გაყოფა. წილადის გაძლიერება და დაქვეითება. წილადის გაერთიანება და დაშლა \dotfill 235 \\
§11. & წილადის გაყოფა. წილადის გაძლიერება და დაქვეითება. წილადის გაერთიანება და დაშლა \dotfill 253 \\
§12. & $y = \frac{k}{x}$, $y = x^2$ და $y = \sqrt{x}$ ფუნქციები. ფუნქციის სიდიდეთა თვისებები \dotfill 268 \\
§13. & ტოლობის მახასიათებლები \dotfill 275 \\
\multicolumn{2}{l}{ტესტი გამოცდისათვის №2} \\
§14. & ტოლობის გადაჭრა \dotfill 294 \\
§15. & არათოლობა და პროპორცია \dotfill 308 \\
§16. & ტოლობის გადაჭრა \dotfill 344 \\
§17. & არათოლობა და ლოგარითმული ფუნქცია, გრაფიკები და უტოლობები \dotfill 359 \\
§18. & არათოლობა \dotfill 375 \\
§19. & პროპორცია და ლოგარითმული ფუნქცია \dotfill 395 \\
§20. & ტესტი გამოცდისათვის №3 \dotfill 404 \\
§21. & მათემატიკა საბაზისო სკოლისათვის \dotfill 422 \\
\end{tabular}
\section*{ნაწილი I}

\subsection*{§ 1. ნატურალური და მთელი რიცხვები}

ნატურალური და მთელი რიცხვები. არითმეტიკული მოქმედებები. ნატურალური რიცხვები, რაოდენობის, სიდიდის დასათვლელად გამოიყენება. ნატურალური რიცხვების რიგი 1 რიცხვიდან იწყება და უსასრულოდ გრძელდება შემდეგნაირად:

\[
1, 2, 3, 4, \ldots
\]

სიდიდეების დასათვლელად, რაოდენობის გამოსახატავად ნულოვანი რიცხვი არ გამოიყენება.

\textbf{მთელი რიცხვები} — ეს არის ნატურალური რიცხვები (\(1, 2, \ldots\)) და მათი \textbf{უარყოფითი წილები} (\(-1, -2, \ldots\)) და \textbf{ნულიც}. მთელი რიცხვების რიგი უსასრულოდ გრძელდება ორივე მიმართულებით:

\[
\ldots, -3, -2, -1, 0, 1, 2, 3, \ldots
\]

მათემატიკაში ხშირად გვხვდება მოქმედებები, რომლებიც ორი რიცხვისგან ქმნიან ახალ რიცხვს. მაგალითად, ორი რიცხვის შეკრება: \(a + b\), გამრავლება \(a b\), სხვაობა \(a - b\), გაყოფა \(\frac{a}{b}\).

ნატურალური რიცხვების \(a, b, c\) შემთხვევაში მოქმედებებს აქვთ შემდეგი თვისებები:

\[
a + b = b + a, \quad (a + b) + c = a + (b + c) \quad \text{(შეკრების კომუტატიურობა და ასოციურობა)}
\]

\textbf{მაგალითი}: შევამოწმოთ ასოციურობის თვისება: \(1925 + 317 + 3075\)

\[
(1925 + 317) + 3075 = (317 + 1925) + 3075 = 317 + (1925 + 3075) = 317 + 5000 = 5317.
\]

\textbf{კომუტატიურობა}: \(a\) რიცხვის გამრავლება \(b\) რიცხვზე იგივეა, რაც \(b\)-ს გამრავლება \(a\)-ზე. ეს ნიშნავს, რომ \(a b = b a\).

\textbf{მაგალითი}: \(17 \cdot 20 = 20 \cdot 17\).

მთელი რიცხვების შემთხვევაში მოქმედებებს აქვთ შემდეგი თვისებები:

\begin{itemize}
\item \textbf{შეკრების კომუტატიურობა}: \(a + b = b + a\)
\item \textbf{გამრავლების ასოციურობა}: \((a b) c = a (b c)\)
\item \textbf{გამრავლების დისტრიბუტიურობა}: \(a (b + c) = a b + a c\)
\end{itemize}

\[
(a - b) c = a c - b c
\]

\textbf{მაგალითი}: შევამოწმოთ დისტრიბუტიურობა: \(1756 \cdot 2179 - 1756 \cdot 2178\)

\[
1756 \cdot 2179 - 1756 \cdot 2178 = 1756 (2179 - 2178) = 1756 \cdot 1 = 1756.
\]

\textbf{გაყოფა}: \(a\) რიცხვის გაყოფა \(b\)-ზე იგივეა, რაც \(a\)-ს გამრავლება \(\frac{1}{b}\)-ზე.

\textbf{ნატურალური რიცხვის ხარისხი}: განისაზღვრება როგორც \(a^n\), სადაც \(a\) რიცხვი \(n\)-ჯერ ნატურალურად გამრავლებულია:

\[
a^n = \underbrace{a \cdot a \cdot \ldots \cdot a}_{n \text{ ჯერ}}.
\]

თუ \(n = 0\), მაშინ \(a^0 = 1\). თუ \(n = 1\), მაშინ \(a^1 = a\). თუ \(n = -1\), მაშინ \(a^{-1} = \frac{1}{a}\).

\subsection*{II. ნატურალური რიცხვების წარწერა ათობით სისტემაში}

ათობით სისტემაში ნატურალური რიცხვები იწერება ციფრებით:

\[
0, 1, 2, 3, 4, 5, 6, 7, 8, 9.
\]

მაგალითად, რიცხვი \(4193\) იწერება როგორც:

\[
4193 = 3 \cdot 10^0 + 9 \cdot 10^1 + 1 \cdot 10^2 + 4 \cdot 10^3.
\]

ანუ:

\[
4193 = 3 + 90 + 100 + 4000.
\]

\subsection*{III. გაყოფადობის ნიშნები}

\textbf{1. რიცხვი \(m\) იყოფა 2-ზე}, თუ მისი ბოლო ციფრი ლუწია (\(0, 2, 4, 6, 8\)).

\textbf{2. რიცხვი \(m\) იყოფა 3-ზე}, თუ მისი ციფრების ჯამი იყოფა 3-ზე.

\textbf{მაგალითი}: \(48 + 64 + 96\) — ციფრების ჯამი \(16 + 19 + 15 = 50\), რაც არ იყოფა 3-ზე.

\textbf{3. რიცხვი \(m\) იყოფა 5-ზე}, თუ მისი ბოლო ციფრი არის \(0\) ან \(5\).

\textbf{4. რიცხვი \(m\) იყოფა 9-ზე}, თუ მისი ციფრების ჯამი იყოფა 9-ზე.

\textbf{5. რიცხვი \(m\) იყოფა 11-ზე}, თუ მისი ციფრების ალტერნატიული ჯამი იყოფა 11-ზე.

\[
\overline{ab} = b + 10a
\]

\[
\overline{abc} = c + 10b + 100a
\]

\textbf{მაგალითი}: \(ab + ba = b + 10a + a + 10b = 11a + 11b = 11(a + b)\), ანუ \(ab + ba\) იყოფა 11-ზე.
\textbf{6. 10-ზე მეტი და მცირეა ის რიცხვი, რომელიც მეტია ორჯერითა 3-ზე.}

\section*{IV. კენტი და ლუწი რიცხვები.}
\textbf{ლუწი} არის ის რიცხვი, რომელიც იყოფა 2-ზე, ანუ ლუწი რიცხვი არის ლუწი მულტიპლიკაცია:
\[
\ldots, -4, -2, 0, 2, 4, 6, \ldots
\]
თუ მოცემულ რიცხვს არ იყოფა 2-ზე, მაშინ იგი არის \textbf{კენტი} რიცხვი. კენტი რიცხვების მაგალითებია:
\[
\ldots, -3, -1, 1, 3, 5, \ldots
\]

თუ მოცემულ რიცხვს ნამრავლი აქვს $n=2k$, სადაც $k$ არის მთელი რიცხვი, მაშინ $n$ არის ლუწი, ხოლო თუ $n=2k+1$, მაშინ $n$ არის კენტი.

\textbf{მაგალითი.} მოცემულია რიცხვი $n=2r+23$. რადგან $2r$ არის ლუწი, ხოლო $23$ არის კენტი, მათი ჯამი $n$ იქნება კენტი.

\textbf{ამოხსნა.} რადგან $2r$ ლუწია, ხოლო $23$ კენტი, მათი ჯამი $n=2r+23$ იქნება კენტი.

\section*{V. მარტივი და შედგენილი რიცხვები.}
\textbf{მარტივი} რიცხვი არის ის ბუნებრივი რიცხვი, რომელიც იყოფა მხოლოდ ორ გამყოფზე: 1 და თავის თავზე. მაგალითები: $2, 3, 5, 7, 11, 13$.

\textbf{შედგენილი} რიცხვი არის ის ბუნებრივი რიცხვი, რომელიც იყოფა ერთზე მეტი გამყოფით, გარდა 1-ისა და თავის თავისა. მაგალითები: $4, 6, 8, 9, 10, 12, 14, 15$.

\textbf{მაგალითი.} $14=2\cdot 7$, $81=3\cdot 3\cdot 3\cdot 3$, $484=2\cdot 2\cdot 11\cdot 11$.

რიცხვის მარტივ გამყოფებად დაშლისას ვიღებთ:
\[
525 \quad 3 \\
175 \quad 5 \\
35 \quad 5 \\
7 \quad 7 \\
1
\]
შედეგად მივიღებთ:
\[
525 = 3 \cdot 5 \cdot 5 \cdot 7 = 3 \cdot 5^2 \cdot 7
\]

\section*{VI. უდიდესი საერთო გამყოფი.}
\textbf{უდიდესი საერთო გამყოფი} არის ის უდიდესი ბუნებრივი რიცხვი, რომელიც იყოფს მოცემულ რიცხვებს.

თუ $D(m,n)=1$, მაშინ $m$ და $n$ არის ურთიერთ მარტივი რიცხვები.

\textbf{მაგალითი.} ვიპოვოთ $D(126;540;630)$.

\textbf{ამოხსნა.} მარტივ გამყოფებად დაშლისას მივიღებთ:
\[
\begin{array}{c|c c c|c c c|c}
126 & 2 && 540 & 2 && 630 & 2 \\
63 & 3 && 270 & 3 && 315 & 3 \\
21 & 3 && 90 & 3 && 105 & 5 \\
7 & 7 && 30 & 5 && 21 & 7 \\
1 &   && 6 & 2 && 3 & 3 \\
  &   && 3 & 3 && 1 &   \\
  &   && 1 &   &&   &  
\end{array}
\]
ანუ:
\[
126 = 2 \cdot 3^2 \cdot 7, \quad 540 = 2^2 \cdot 3^3 \cdot 5, \quad 630 = 2 \cdot 3^2 \cdot 5 \cdot 7
\]
საერთო გამყოფებია $2$, $3$, $3$, ანუ:
\[
D(126;540;630) = 2 \cdot 3 \cdot 3 = 18
\]

\section*{VII. უმცირესი საერთო ჯერადი.}
\textbf{უმცირესი საერთო ჯერადი} არის ის უმცირესი ბუნებრივი რიცხვი, რომელიც იყოფა მოცემულ რიცხვებზე.

\textbf{მაგალითი.} ვიპოვოთ $K(270;300;315)$.

\textbf{ამოხსნა.} მარტივ გამყოფებად დაშლისას მივიღებთ:
\[
\begin{array}{c|c c c|c c c|c}
270 & 2 && 300 & 2 && 315 & 3 \\
135 & 3 && 150 & 3 && 105 & 5 \\
45 & 3 && 50 & 5 && 21 & 7 \\
15 & 5 && 10 & 2 && 3 & 3 \\
5 & 5 && 5 & 5 && 1 &   \\
1 &   && 1 &   &&   &  
\end{array}
\]
ანუ:
\[
270 = 2 \cdot 3^3 \cdot 5, \quad 300 = 2^2 \cdot 3 \cdot 5^2, \quad 315 = 3^2 \cdot 5 \cdot 7
\]
ამრიგად:
\[
K(270;300;315) = 2^2 \cdot 3^3 \cdot 5^2 \cdot 7 = 18900
\]

\section*{VIII. ნაშთი. ნაშთის არსებობის პირობები.}
თუ $m$ და $n$ არ არის ურთიერთ მარტივი რიცხვები, მაშინ $m$-ის ნაშთების სისტემა შეიძლება არ ჰქონდეს გადაწყვეტა.

მაგალითად, $n=28=2^2 \cdot 7$, $m=8=2^3$. თუ $m$-ის ნაშთი $a$ და $n$-ის ნაშთი $b$ არ აკმაყოფილებს გარკვეულ პირობებს, სისტემა არ გადაწყდება.

\textbf{მაგალითი.} ვიპოვოთ $8x \equiv 5 \ (\text{mod} \ 7)$.

\textbf{ამოხსნა.} $8 \equiv 1 \ (\text{mod} \ 7)$, ამიტომ $x \equiv 5 \ (\text{mod} \ 7)$.

\textbf{მაგალითი.} ვიპოვოთ $777783 - 6$.

\textbf{ამოხსნა.} $777783 - 6 = 777777$.

რიცხვი $777777$ იყოფა 7-ზე, რადგან $1+2+3+4+5+6=21$ და $21$ იყოფა 7-ზე.

\textbf{მაგალითი.} რა არის $137^{100}$-ის ბოლო ციფრი?

\textbf{ამოხსნა.} $137^{100}$-ის ბოლო ციფრი იგივეა, რაც $7^{100}$-ის ბოლო ციფრი. $7$-ის ხარისხების ბოლო ციფრები მეორდება ყოველ 4 ნაბიჯში: $7, 9, 3, 1$. $100$ იყოფა 4-ზე, ამიტომ $137^{100}$ მთავრდება 1-ით.
\section*{IX. თულის ორბიტის სისტემა, ათობითიდან ორობითში გადაყვანა და პირიქით}

როდესაც ფული, ელექტრონული სიგნალი, ან სხვა ჩაწერილი ინფორმაცია (მათ შორის ტექსტი) ერთეულ 10-ზეა წარმოდგენილი, მაგალითად, თულის ორბიტის სიგრძე $100^{\circ}$, საჭიროა მისი გადაყვანა სხვა ერთეულში. თუ ათობითი სისტემა 10-ზეა დაფუძნებული, ორობითი სისტემა 2-ზეა დაფუძნებული. 

\[
\begin{array}{c|c}
100 & 2 \\
50 & 2 \\
25 & 2 \\
12 & 2 \\
6 & 2 \\
3 & 2 \\
1 & 2 \\
0 & \\
\end{array}
\]

მიღებულია $(1100100)_2$ ანუ 100-ის ორობითი ჩანაწერი:
\[
100 = (1100100)_2
\]

მაგალითად, $(101)_2$, $(10010)_2$ არ არის ათობითში გადაყვანილი, ხოლო $(10201)_2$ ან $(2031)_2$ არ არის სწორი ორობითი ჩანაწერი.

ათობითიდან ორობითში გადასაყვანად საჭიროა რიცხვი გავყოთ 2-ზე და შევინახოთ ნაშთები, სანამ განაყოფი არ გახდება 0.

მაგალითად:
\[
1029 = 1 \cdot 10^3 + 0 \cdot 10^2 + 2 \cdot 10^1 + 9 \cdot 10^0 = 1000 + 20 + 9
\]

ეს პროცესი შეიძლება გაგრძელდეს სხვა სისტემებშიც, თუ თულის ორბიტის სიგრძე წარმოდგენილია სხვა სისტემაში.

ორობითიდან ათობითში გადასაყვანად საჭიროა თითოეული ციფრის მნიშვნელობა გავამრავლოთ შესაბამისი 2-ის ხარისხზე.

მაგალითად:
\[
(10011)_2 = 1 \cdot 2^4 + 0 \cdot 2^3 + 0 \cdot 2^2 + 1 \cdot 2^1 + 1 \cdot 2^0 = 16 + 2 + 1 = 19
\]

\textbf{მაგალითი.} რომელია მეტი, $35$ თუ $(110111)_2$?

\textbf{ამოხსნა.} $(110111)_2 = 1 \cdot 2^5 + 1 \cdot 2^4 + 0 \cdot 2^3 + 1 \cdot 2^2 + 1 \cdot 2^1 + 1 \cdot 2^0 = 32 + 16 + 4 + 2 + 1 = 53$, ამიტომ $(110111)_2 > 35$.

\textbf{მაგალითი.} თუ ჩვენ გვაქვს მექანიკური გონება, რომელიც 1-ს და 1000-ს შორის, შეადარებს ორ რიცხვს, შევძლებთ თუ არა 10-იან სისტემაში გადაყვანას?

\textbf{ამოხსნა.} თუ გვაქვს $(110111)_2$ და $(100011)_2$, შევადარებთ მათ ორობითში: $(110111)_2 > (100011)_2$, რადგან 2-ის სისტემაში პირველი ციფრი 1-ია, ხოლო მეორე ციფრი 0-ია. ეს ნიშნავს, რომ $(110111)_2$ უფრო დიდია.

\section*{§ 2. რაციონალური რიცხვები}

რაციონალური რიცხვები არიან, რომლებიც შეიძლება გამოიხატოს ორი მთელი რიცხვის ტოლობითი ნამრავლით. განვმარტოთ ტერმინი: \textbf{მარტივი წილადი} $m : d$ (ანუ $\frac{m}{d}$), სადაც $m$ არის მთლიანი და $d$ ნატურალური, ანუ რაციონალური რიცხვი.

რაციონალური რიცხვების მაგალითებია: $\frac{2}{5}$, $\frac{5}{2}$, $-\frac{3}{4}$ და სხვ. მათ შეიძლება ჰქონდეთ დადებითი ან უარყოფითი მნიშვნელობა.

რაციონალური რიცხვების კლასიფიკაცია:
\begin{enumerate}
\item წილადი;
\item შერეული რიცხვები;
\item ათწილადი.
\end{enumerate}

\subsection*{I. წილადი}
წილადი (მარტივი წილადი) არის $\frac{m}{d}$, სადაც $m$ და $d$ ნატურალური რიცხვებია. მაგალითად, $\frac{2}{5}$ ნიშნავს, რომ მთელი გაყოფილია 5 თანაბარ ნაწილად და აღებულია 2 ნაწილი.

\textbf{მარტივი წილადი} შეიძლება იყოს დადებითი ან უარყოფითი. მაგალითად, $-\frac{2}{5}$ ნიშნავს, რომ აღებულია 2 ნაწილი 5 თანაბარი ნაწილიდან, მაგრამ უარყოფითი მიმართულებით.

\textbf{შერეული რიცხვი} შედგება მთელი და წილადი ნაწილისგან. მაგალითად, $1 \frac{1}{4}$ ნიშნავს ერთ მთლიანს და დამატებით $\frac{1}{4}$-ს.

ნატურალური რიცხვი $n$ არის მთელი. როდესაც წილადი ნაწილის მნიშვნელობა ნაკლებია ერთზე, შერეული რიცხვი შეიძლება გადაიქცეს მარტივ წილადში.

მაგალითად:
\[
1 \frac{1}{4} = \frac{5}{4}
\]
\[
\frac{2}{4} = \frac{1}{2}
\]

\begin{center}
\begin{tikzpicture}
\draw[->] (0,0) -- (5,0);
\draw (0,0) -- (1,0);
\draw (1,0) -- (2,0);
\draw (2,0) -- (3,0);
\draw (3,0) -- (4,0);
\draw (4,0) -- (5,0);
\draw (2,0.5) circle (0.5);
\node at (2,0.5) {$\frac{1}{4}$};
\node at (3,0.5) {$\frac{1}{4}$};
\node at (2.5,1) {$\frac{2}{4}$};
\end{tikzpicture}
\end{center}

ზოგადად, თუ $\frac{n}{d}$ და $\frac{m}{b}$ წილადი რიცხვების ტოლობაა, მაშინ $nb = md$.

თუ წილადი ტოლია, მაშინ მათი მნიშვნელობები თანაბარია.

\textbf{მაგალითი.} $\frac{1}{2}$ და $\frac{2}{4}$ ტოლია, რადგან $1 \cdot 4 = 2 \cdot 2$.

\textbf{გადაყვანა.} $\frac{10}{20}$ და $\frac{8}{20}$ შეიძლება გადაიქცეს მარტივ წილადში.

\[
\frac{10}{20} = \frac{1}{2}, \quad \frac{8}{20} = \frac{2}{5}
\]

\textbf{გადაყვანა.} $\frac{6}{48}$ და $\frac{8}{48}$.

\textbf{წილადის შედარება.} ტოლმნიშვნელოვანი წილადის შედარება ხდება მათი მნიშვნელობის მიხედვით. მაგალითად:
\[
\frac{6}{9} > \frac{5}{9}
\]

\textbf{მაგალითი.} $\frac{8}{17}$ და $\frac{8}{19}$ შედარებისას $\frac{8}{17} > \frac{8}{19}$.

\textbf{მაგალითი.} შედარება წილადების:
\[
\frac{2004}{2005} \quad \text{და} \quad \frac{2005}{2006}
\]
\textbf{ამოხსნა.} $\frac{2004}{2005} < \frac{2005}{2006}$, რადგან პირველი წილადი მცირეა.

\textbf{მაგალითი.} შედარება:
\[
\frac{14}{29} \quad \text{და} \quad \frac{19}{37}
\]
\textbf{მაგალითი.} პარალელური წრედი ნაშთია $\frac{1}{2}$-ზე, ხოლო მეორე წრედი $\frac{1}{d}$-ზე. ესე იგი პირველი წრედი არის $n/d$, ხოლო მეორე წრედი არის $-n/d$. 

\textbf{ურთიერთსაწინააღმდეგო წრედები.} ორი წრედი ურთიერთსაწინააღმდეგოა, თუ მათი ჯამი უდრის ნულს. მაგალითად, თუ $n/d$ არის წრედი, მაშინ მისი ურთიერთსაწინააღმდეგო წრედი არის $-n/d$, რადგანაც
\[
\frac{n}{d} + \left( -\frac{n}{d} \right) = \frac{n}{d} - \frac{n}{d} = 0.
\]

რადგანაც წრედის ხაზი გაბათილებულია, ამიტომ $-n/d$ წრედის საწინააღმდეგო არის:
\[
-\frac{n}{d} = \frac{-n}{d} = \frac{n}{-d}.
\]

\textbf{წრედების შეკრება და გამოკლება.} მაგალითად, თუ შევკრიბოთ $5$ ერთნაირი მნიშვნელიანი წრედი, მათი ჯამი მიიღება მნიშვნელი უცვლელად, ხოლო მახასიათებლების შეკრებით:
\[
\frac{3}{5} + \frac{4}{5} = \frac{3+4}{5}, \quad \frac{2}{7} + \frac{5}{7} = \frac{2+5}{7}, \quad \frac{2}{7} - \frac{5}{7} = \frac{2-5}{7}, \quad \frac{-3}{7} - \frac{3}{7} = \frac{-3-3}{7}.
\]

თუ ორი წრედი განსხვავებული მნიშვნელიანი აქვთ, მაშინ მათი შეკრება ხდება საერთო მნიშვნელიანი და მახასიათებლების შეკრებით:
\[
\frac{3}{5} + \frac{4}{7} = \frac{21}{35} + \frac{20}{35} = \frac{41}{35}, \quad \frac{2}{3} + \frac{5}{4} = \frac{8}{12} + \frac{15}{12} = \frac{23}{12}.
\]

\textbf{წრედების გამრავლება და გაყოფა.} ამ შემთხვევაში მახასიათებლები მრავლდება ერთმანეთზე, ხოლო მნიშვნელიებიც მრავლდება ერთმანეთზე:
\[
\frac{2}{5} \cdot \frac{4}{7} = \frac{8}{35}, \quad \frac{6}{9} \cdot \frac{2}{1} = \frac{12}{9}, \quad \frac{6}{2} \cdot \frac{4}{9} = \frac{24}{18}.
\]

თუ დადებით წრედს გავამრავლებთ უარყოფითზე, ნიშანი იქნება უარყოფითი:
\[
\frac{2}{7} \cdot \left( -\frac{4}{7} \right) = -\frac{8}{49}.
\]

თუ უარყოფით წრედს გავამრავლებთ უარყოფითზე, ნიშანი იქნება დადებითი:
\[
\left( -\frac{2}{5} \right) \cdot \left( -\frac{4}{9} \right) = \frac{8}{45}.
\]

მაგალითად, რომ ერთ წრედს გავყოთ მეორეზე, უნდა გავამრავლოთ პირველი წრედი მეორე წრედის შებრუნებულზე:
\[
\frac{2}{5} : \frac{4}{7} = \frac{2}{5} \cdot \frac{7}{4} = \frac{14}{20}.
\]

---

თუ $n/d$ წრედის შებრუნებული წრედი არის $d/n$, მაშინ $n$ და $d$ არანულოვანი რიცხვებია.

\section*{II. წილადი და არაწილადი რიცხვები}
\textbf{შერეული რიცხვები.} რიცხვი, რომელიც შედგება მთელი რიცხვისა და წილადისგან, მაგალითად $\frac{8}{3}$, არის შერეული რიცხვი. 

მაგალითად:
\[
3 \frac{2}{3}, \quad 5 \frac{2}{3}
\]

\textbf{მაგალითი.} რომ შევკრიბოთ შერეული რიცხვები და წილადი ნაშთიანად:
\[
5 \frac{2}{3} = \frac{17}{3}, \quad 7 \frac{1}{3} = \frac{22}{3}, \quad -3 \frac{4}{9} = -\frac{31}{9}.
\]

შერეული რიცხვების შეკრებისას, მაგალითად $8 \frac{2}{3}$ და $8 \frac{2}{3}$:
\[
8 \frac{2}{3} + 8 \frac{2}{3} = 16 + \frac{4}{3} = 17 \frac{1}{3}.
\]

თუ დავამატებთ შერეულ რიცხვებს, ჯერ ვამატებთ მთელ რიცხვებს, შემდეგ წილადის ნაწილებს:
\[
22 = 3 + 7 + 1, \quad 26 = 3 + 8 + 2.
\]

თუ შერეული რიცხვი ნიშნის მიხედვით არის უარყოფითი, წილადის ნაშთი ნიშანს ინარჩუნებს:
\[
3 \frac{5}{3} + 1 \frac{1}{3} = 5 + \frac{1}{3}, \quad 27 \frac{4}{4} + 6 \frac{3}{4} = 33 \frac{3}{4}.
\]

მაგალითად:
\[
2 \frac{1}{3} - 6 \frac{2}{6} = -4 \frac{1}{6}.
\]

---

\section*{III. ათწილადი}
ყველა რიცხვი, რომლის წილადის მნიშვნელი არის $10$-ის ხარისხი, არის ათწილადი. 

მაგალითად:
\[
0,327 = \frac{327}{1000} = \frac{3}{10} + \frac{2}{100} + \frac{7}{1000}
\]
\[
0,0327 = \frac{327}{10000} = \frac{3}{100} + \frac{2}{1000} + \frac{7}{10000}
\]
\[
1,52 = 1 + \frac{5}{10} + \frac{2}{100}
\]
\[
36,7 = 36 + \frac{7}{10}
\]

\textbf{ათწილადის შეკრება და გამოკლება.} მაგალითად:
\[
0,138 + 3,251 = 3,389, \quad 53,2700 - 7,6512 = 45,6188, \quad 16,300 - 4,752 = 11,548.
\]

\textbf{ათწილადის გამრავლება.} მაგალითად:
\[
3,07 \times 1,2 = 3,684
\]
(2 ციფრი მარჯვნივ).

\textbf{ათწილადის გაყოფა.} მაგალითად:
\[
39,16 : 4 = 9,79, \quad 1,047 : 3 = 0,349.
\]
\[
\begin{array}{r}
36 \\
31 \\
28 \\
36 \\
36 \\
0
\end{array}
\quad
\begin{array}{r}
9 \\
14 \\
12 \\
27 \\
27 \\
0
\end{array}
\]

\textbf{ათწილადის გაყოფა.} როგორც რიცხვი (კასრიც), ისე მისი ათწილადი შეიძლება გავყოთ სხვა რიცხვზე (მთელზე). ჯერ ვყოფთ მარჯვენა რიცხვს მარცხენა რიცხვში ისე, როგორც მთელ რიცხვებს ვყოფთ. შემდეგ ვსვამთ წერტილს და ვაგრძელებთ გაყოფას. მაგალითად:

\[
8,06 : 3,1 = 80,6 : 31 = 2,6, \quad 12,5 : 0,16 = 1250 : 16 = 78,125
\]

\[
\frac{62}{186} \quad \frac{186}{0}
\]

ათწილადის $10^n$-ზე, $n \in \mathbb{N}$, გამრავლებისას მისი წერტილი n ერთეულით მარჯვნივ გადაინაცვლებს. მაგალითად:

\[
0,258 \cdot 100 = 25,8, \quad 0,3189 \cdot 1000 = 318,9,
\]
\[
2,318 \cdot 100 = 231,8, \quad 59,128 : 10 = 5,9128.
\]

\textbf{რიცხვის გარდაქმნა ათწილადად, უსასრულო პერიოდულ ათწილადად.} რიცხვის გარდაქმნა ათწილადად ნიშნავს, რომ იგი უნდა გავყოთ მნიშვნელი. მაგალითად:

\[
\frac{7}{25} = 7 : 25 = 0,28
\]

რიცხვის მნიშვნელი თუ შეიცავს მხოლოდ 2-ისა და 5-ის ხარისხებს, იგი გარდაიქმნება საბოლოო ათწილადად, ანუ უსასრულო ათწილადი არ იქნება. თუ მნიშვნელი შეიცავს სხვა მარტივ რიცხვებსაც, მაშინ მივიღებთ უსასრულო პერიოდულ ათწილადს. მაგალითად:

\[
\frac{1}{3} = 0,333\ldots, \quad \frac{7}{9} = 0,777\ldots, \quad \frac{6}{11} = 0,545454\ldots
\]

პერიოდი შეიძლება იყოს ერთი ციფრი ან რამდენიმე ციფრი. მაგალითად:

\[
\frac{1}{3} = 0,(3), \quad \frac{7}{9} = 0,(7), \quad 4,042424\ldots = 4,(042)
\]

შენიშვნა: უსასრულო პერიოდული ათწილადის გარდაქმნა კასრად შეიძლება გაკეთდეს შემდეგი მეთოდით: ვნიშნავთ $x$-ს ათწილადით და ვამრავლებთ $10^n$-ზე, სადაც $n$ არის პერიოდის სიგრძე, შემდეგ ვაკლებთ პირვანდელ $x$-ს და ვიღებთ მთელ რიცხვს, რომლის გაყოფითაც ვპოულობთ კასრს.

მაგალითად:
\[
0,(3) = \frac{1}{3}, \quad 0,(54) = \frac{54}{99}, \quad \frac{6}{11}, \quad 3,(173) = \frac{3173 - 31}{990} = \frac{1571}{495}
\]

\section*{IV. მთელ რიცხვების და ათწილადების დამრგვალება}
განვიხილოთ შემდეგი მაგალითი: რიცხვების დამრგვალება ხდება ისე, რომ მივიღოთ ახლოს მდგომი რიცხვი. მაგალითად, თუ გვაქვს 57238 და გვინდა დავამრგვალოთ ათასეულამდე, ვნახავთ ასეულების ციფრს (2), რადგან იგი ნაკლებია 5-ზე, ათასეულების ციფრი არ იცვლება და მივიღებთ 57000.

თუ ასეულების ციფრი მეტია ან ტოლია 5-ზე, ათასეულების ციფრი იზრდება ერთით. მაგალითად, 57675 დამრგვალებული ათასეულამდე იქნება 58000.

თუ გვინდა რიცხვის დამრგვალება ათეულამდე, ვნახავთ ერთეულების ციფრს. თუ იგი ნაკლებია 5-ზე, ათეულების ციფრი არ იცვლება, ხოლო ერთეულები ნულდება. მაგალითად, 7628 დამრგვალებული ათეულამდე იქნება 7630.

თუ ერთეულების ციფრი მეტია ან ტოლია 5-ზე, ათეულების ციფრი იზრდება ერთით. მაგალითად, 7604 დამრგვალებული ათეულამდე იქნება 7600.

მნიშვნელოვანია ვიცოდეთ და მისი დამრგვალების მეთოდიც, რასაც \textbf{მათემატიკური წოლა} ეწოდება. ეს ნიშნავს, რომ რიცხვი დამრგვალდება ისე, რომ ნაშთი იყოს მინიმალური. მაგალითად, 9675≈9700 (რადგან 9675 უფრო ახლოსაა 9700-ზე).

დამრგვალების მაგალითები:
682,51 დამრგვალებული ასეულამდე იქნება 700, ხოლო ათეულამდე — 680. მაგალითად:
\[
682,51 \approx 700, \quad 682,51 \approx 680
\]

თუ გვინდა რიცხვის დამრგვალება ვერტიკალურად, მაგალითად, თუ 4,738-ს დავამრგვალებთ მეათედამდე, მივიღებთ 4,7.

\textbf{შენიშვნა:} დამრგვალებისას უნდა ვიცოდეთ:
1) დამრგვალების მიზანი და რა სიზუსტით გვინდა მივიღოთ რიცხვი;
2) თუ დამრგვალება ხდება მარცხნივ 5-ით, მაშინ რიცხვი იზრდება 1-ით, ხოლო თუ ნაკლებია 5-ზე, არ იცვლება.

მაგალითად:
\[
31,967 \approx 31,97 \quad \text{(დამრგვალება მეათედამდე)}
\]
\[
0,653 \approx 0,7 \quad \text{(დამრგვალება მეათედამდე)}
\]

თუ თვლების დამრგვალების წესი ცვლის რიცხვს ისე, რომ იგი ახლოს იყოს ნულთან, მაშინ მივიღებთ ნულს. მაგალითად, თუ რიცხვი 0,032 დამრგვალდება მეათედამდე, მივიღებთ 0.

მაგალითად:
\[
13,5203 \approx 13,520 \quad \text{(დამრგვალება მეათედამდე)}
\]
\[
3,027 \approx 3,0 \quad \text{(დამრგვალება მეათედამდე)}
\]
\[
31,967 \approx 32,0 \quad \text{(დამრგვალება მეათედამდე)}
\]

\end{document}