% !TEX TS-program = xelatex
\documentclass[12pt,oneside]{book}

\usepackage[a4paper,margin=1in]{geometry}

\usepackage{fontspec}
\defaultfontfeatures{Ligatures=TeX,Scale=MatchLowercase}
\usepackage{polyglossia}
\setdefaultlanguage{georgian}
\setotherlanguage{english}

% Georgian fonts - DejaVu Serif supports Georgian
\setmainfont{DejaVu Serif}
\newfontfamily\georgianfont{DejaVu Serif}[Script=Georgian]
\setsansfont{DejaVu Sans}
\setmonofont{DejaVu Sans Mono}

% Math: STIX Two Math
\usepackage{unicode-math}
\setmathfont{STIX Two Math}

\usepackage[protrusion=true,expansion=false]{microtype}
\usepackage{enumitem}
\setlist{nosep}
\usepackage{graphicx}
\usepackage[unicode]{hyperref}

\usepackage{fancyhdr}
\pagestyle{fancy}
\fancyhf{}
\fancyhead[LE,RO]{\thepage}
\fancyhead[LO,RE]{\nouppercase{\leftmark}}
\renewcommand{\headrulewidth}{0pt}
\renewcommand{\chaptermark}[1]{\markboth{#1}{}}
\renewcommand{\sectionmark}[1]{\markright{#1}}

% Georgian mathematical notations
\newcommand{\Z}{\mathbb{Z}}
\newcommand{\dotsep}{\,,\,}


\begin{document}

\begin{center}
{\Large \textbf{ნაწილი II}}\\[0.25em]
\textbf{\S 1. ნატურალური და მთელი რიცხვები}
\end{center}

\section*{სავარჯიშოების ნაკრები A}

\begin{enumerate}[label=\textbf{\arabic*.\arabic*.}, leftmargin=2.2em, itemsep=0.35em]

% ---------- გვერდი 1 ----------
\item რომელი შემდეგი რიცხვებიდან არის მარტივი?
\\(1) 55\quad (2) 617\quad (3) 5027\quad (4) 42028

\item რომელი შემდეგი რიცხვებიდან არის შედგენილი?
\\(1) 389\quad (2) 1507\quad (3) 323061\quad (4) 52713

\item განსაზღვრეთ შემდეგი რიცხვების ლუწობა (ლუწი/კენტი):
\\(1) 531\quad (2) 2399\quad (3) 4057\quad (4) 20712

\item აირჩიეთ სწორი დებულებები მთელ რიცხვებზე:
\\(1) და (2) \emph{[დებულების ვარიანტები წყაროში]}\\
(3) \emph{შესაძლებელია ჩამოვაყალიბოთ \ldots}\quad
(4) \emph{ნული არც დადებითია და არც უარყოფითი.}

\item განსაზღვრეთ თითოეული შედეგის ნიშანი (დადებითი/უარყოფითი/ნული) ზუსტი მნიშვნელობის გამოთვლის \emph{გარეშე}.
\\(კონტექსტი: ჯამები/ნამრავლები მითითებული ნიშნებით.)

\item რომელი შემდეგთაგანია მთელი რიცხვები?
\\(1) $a+b$\quad (2) $a-b$\quad (3) $c+d$\quad (4) $\tfrac{c}{d}$\quad
(5) $cd$\quad (6) $a+c$\quad (7) $a-c$\quad (8) $ac$\quad (9) $2c+a$\quad
(10) $2c-3a$\quad (11) $a^2-c^2$\quad (12) $3a^2+2c^2$

\item გამოთვალეთ (ოპერაციების თანმიმდევრობა და ფრჩხილები):
\\(1) $24$\quad (2) $77$\quad (3) $128$\quad (4) $4128$ \hfill\emph{[მოკლე პრაქტიკის ამოცანები]}

\item იპოვეთ შემდეგი წყვილების (ან სიმრავლეების) უმცირესი საერთო ჯერადი (უსჯ).
\\(1) 28\quad (2) 99\quad (3) 200\quad (4) 532

\item იპოვეთ შემდეგი მთელი რიცხვების უდიდესი საერთო გამყოფი (უსგ).
\\(1) 12\quad (2) 13\quad (3) 50\quad (4) 120

\item რამდენი ნატურალური გამყოფი აქვს $66$-ს?

\item იპოვეთ მოცემული რიცხვების საერთო გამყოფების რაოდენობა.
\\(1) 15\quad (2) 19\quad (3) 27\quad (4) 48

\item გამოთვალეთ თითოეული წყვილის უსჯ:
\\(1) $12$ და $42$\quad (2) $240$ და $360$\quad (3) $1680$ და $4200$\quad (4) $180,\ 270,\ 450$

\item გამოთვალეთ თითოეული ჯგუფის უსგ:
\\(1) $12$ და $15$\quad (2) $24,\ 30,\ 60$\quad (3) $7,\ 8,\ 13$\quad (4) $20,\ 65,\ 125$

\item გაამარტივეთ შემდეგი გამოსახულებები:
\\(1) $2^5\cdot 3^7\cdot 5^{-3}$ და $2^1\cdot 3^3\cdot 5^2$
\\(2) $2^{-7}\cdot 3^2$ და $2^2\cdot 7^3$ \hfill\emph{[სტანდარტიზებული ხარისხები წყაროს მიხედვით]}

% ---------- გვერდი 2 ----------
\setcounter{enumi}{15}

\item სიტყვიერი ამოცანები გაყოფადობაზე (პარაფრაზირებული სკანირებიდან):
\\(1) თუ რიცხვი ტოვებს ნაშთს $n$ გაყოფისას $m$-ზე, რა არის \ldots ნაშთი?
\\(2) თუ მთელი რიცხვი იზრდება $+2$-ით, როგორ იცვლება ლუწობა?
\\(3) \emph{[გაყოფადობა ჯამებით]} \quad
(4) თუ მთელი რიცხვი იზრდება $+1$-ით, როგორ \ldots?

\item ააგეთ უმცირესი (ან უდიდესი) $n$-ნიშნა რიცხვი, რომელიც აკმაყოფილებს მოცემულ ციფრულ შეზღუდვებს. \emph{[როგორც წყაროში]}

\item რამდენი მთელი რიცხვი აკმაყოფილებს მოცემულ უტოლობას/ინტერვალს? \emph{[მთელი ამონახსნების რაოდენობა]}

\item კლასის ამოცანა დაჯდომა/განლაგების შესახებ ჯამებით 30, 40 და 47. განსაზღვრეთ რამდენი \ldots \emph{[როგორც წყაროში]}

\item არითმეტიკა შერეული ოპერაციებით (აირჩიეთ სწორი შედეგი):
\\(1) $1*00$\quad (2) $25*8$\quad (3) $4*1$\quad (4) $*888$
\\\emph{[აქ $*$ აღნიშნავს თანდასმას წყაროში; გადააკეთეთ საჭიროების მიხედვით კლასში.]}

\item განსაზღვრეთ ხაზგასმული ციფრის ადგილობრივი მნიშვნელობა თითოეულ რიცხვში:
\\(1) $\underline{54312}$ \quad (2) $4532\underline{\phantom{0}}$ \quad (3) $32*25$ \quad
(4) $*3260$ \quad (5) $423*0$ \quad (6) $*2310$ \emph{[პოზიციები წყაროს მიხედვით]}

\item ჩაწერეთ შემდეგი დიდი რიცხვები დაჯგუფებით (ინტერვალებით) და სიტყვებით:
\\(1) 1275\quad (2) 33333\quad (3) 10203040\quad (4) 1919191919

\item ორი სიტყვიერი ამოცანა, რომელიც მოიცავს დრო/სიჩქარის სცენარებს „5 დღე" და „7 დღე" (უსჯ/უსგ გამოყენებები). \emph{[როგორც წყაროში]}

\item თუ $a$ და $b$ არის მთელი რიცხვები და $a-b$ იყოფა $5$-ზე, განსაზღვრეთ, რომელი გამოსახულებები იყოფა $5$-ზე. \emph{[წყაროს მიხედვით]}

\item თუ მთელი რიცხვი იყოფა $17$-ზე და სხვა დაკავშირებული რაოდენობა უდრის $8$-ს, რას შეიძლება ითქვას $a+b$ შესახებ მოდულო $17$? \emph{[წყაროს მიხედვით]}

\item მოკლე პასუხები:
\\(1) \emph{აირჩიეთ სწორი ვარიანტი ნიშნის/ზომის საფუძველზე.}
\\(2) დამრგვალების/შეფასების ამოცანა უახლოეს 10-მდე; შემდეგ უახლოეს 100-მდე. \emph{[როგორც წყაროში]}

\item დაალაგეთ შემდეგი მთელი რიცხვები რიცხვით ხაზზე:
\\(1) $-2$-დან $0$-მდე \quad (2) $-2$-დან $3$-მდე \quad (3) $-5$-დან $1$-მდე \quad (4) $-1$-დან $9$-მდე

\item გამოთვალეთ:
\\(1) $-2+5-6$\quad (2) $-11-(-3)$\quad (3) $14-(-4)$\quad (4) $-17+(+2)$\\
(5) $(-2-3)+2+5$\quad (6) $(-2+6)+(-6)$\quad (7) $(-2-5)-(+3-7)$\quad (8) $(-10+3-4)(4-6)$

\item გამოთვალეთ:
\\(1) $3(9-11)-4(8-9)$ \quad (2) $-5(-7-5)+3(6-9)$ \quad (3) $\dfrac{5\,(9-13)\,(-4)}{-8\,(9-14)}$ \quad (4) $\dfrac{-7\,(20-11)\cdot 2}{3\,(2-16)}$

% ---------- გვერდი 3 ----------
\setcounter{enumi}{30}

\item გამოთვალეთ:
\\(1) $36:3-9$ \quad (2) $36:(3-9)$ \quad (3) $30-15:3$ \quad (4) $(30-15):3$\\
(5) $240-25:(13-8)$ \quad (6) $120:(10-16)-20$ \quad (7) $50-(-20):5$ \quad (8) $(50-(-20)):5$

\item სიტყვიერი ამოცანები:
\\(1) მუშაკი ხარჯავს 6 დღეს ერთ ამოცანაზე და 70 გვერდს მეორეზე; თუ 50 გვერდი არის \ldots განსაზღვრეთ რამდენი დღე სჭირდება \ldots
\\(2) რიცხვი $152\;2\underline{\phantom{0}}3$ იყოფა 16-ზე; იპოვეთ დაკარგული ციფრი. \emph{[როგორც წყაროში]}

\item კლასში A არის 10-ით მეტი მოსწავლე, ვიდრე კლასში B. თუ \ldots როდესაც კლასები გაერთიანდება \ldots იპოვეთ A-ის და B-ის ზომები. \emph{[როგორც წყაროში]}

\item მართკუთხედი პერიმეტრის ინფორმაციით; იპოვეთ გვერდების სიგრძეები მოცემული კავშირებით და იმით, რომ პერიმეტრი არის 28. \emph{[როგორც წყაროში]}

\item მატარებელი გადის 59 კმ პირველ საათში და \ldots განსაზღვრეთ საერთო მანძილი 10 საათის შემდეგ. \emph{[როგორც წყაროში]}

\item დავუშვათ $x,y,z$ არის მთელი რიცხვები. თუ $xy-3z$ იყოფა \ldots განსაზღვრეთ არის თუ არა \ldots

\item აირჩიეთ სწორი სამეცნიერო ნოტაცია (ან სიდიდე) თითოეული რიცხვისთვის:
\\(1) $10$ \quad (2) $10\,000$ \quad (3) $41$

\item გამოთვალეთ კვადრატები (შაბლონის ამოცნობა):
\\(1) $(11)^2$ \quad (2) $(111)^2$ \quad (3) $(1000)^2$ \quad (4) $(10000)^2$

\end{enumerate}

\bigskip
\section*{საკონტროლო ტესტი N1 (A)}

\begin{enumerate}[label=\textbf{\arabic*.}, leftmargin=2.2em, itemsep=0.35em, resume*]

\item რომელი შემდეგი რიცხვებიდან იყოფა $6$-ზე?
\\(a) 273 \quad (b) 412 \quad (c) 1002 \quad (d) 3004

\item გამოთვალეთ: $400:(25-75)-200:(13-33)$
\\(a) $2$ \quad (b) $18$ \quad (c) $-18$ \quad (d) $-2$

\item რომელ წელს 2011-სა და \ldots შორის ხდება \ldots რაოდენობა გაყოფადი $5$-ზე?
\\(a) 2 \quad (b) 5 \quad (c) 3

\item დავუშვათ $a,b,c,d$ განისაზღვრება $a=3\cdot(5-7)$, $b=3-5-7$, $c=ab$, $d=a+b$. აირჩიეთ სწორი თანმიმდევრობა:
\\(a) $a,c,b,d$ \quad (b) $c,a,d,b$ \quad (c) $c,d,b,a$ \quad (d) $c,d,a,b$

\item რიცხვი $123$ იზრდება 8-ით და შემდეგ მცირდება 3-ით. რომელ ინტერვალში მოექცევა შედეგი?
\\(a) 10 \quad (b) 15 \quad (c) 20 \quad (d) 12

\item სკოლაში არის 155 პირველკლასელი და 62 \ldots რომელია უახლოესი მარტივი რიცხვი?
\\(a) 20 \quad (b) 25 \quad (c) 27 \quad (d) 31

\item რომელი რიცხვია $15$-ის ჯერადი?
\\(a) 5005 \quad (b) 4005 \quad (c) 1015 \quad (d) 2015

\item იპოვეთ $101$-ის უახლოესი მარტივი რიცხვი მარჯვნიდან და მარცხნიდან.
\\(a) 909 \quad (b) 816 \quad (c) 807 \quad (d) 809

\item აირჩიეთ სწორი ფაქტორიზაცია:
\\(a) $5$-რაღაც \quad (b) \emph{[ვარიანტები წყაროს მიხედვით: ორწევრიანი vs სამწევრიანი ფაქტორიზაცია]}

\item რა არის $36$-ის უმცირესი გამყოფი?
\\(a) 1 \quad (b) 2 \quad (c) 3 \quad (d) 4

\item წრფივი გამოსახულება ფორმით $2n+3$. რომელი შემდეგთაგანია შესაძლო მნიშვნელობები?
\\(a) $6n+1$ \quad (b) $6n+3$ \quad (c) $3n+3$ \quad (d) $6n-1$

\item რამდენი ნული არის $N$-ის ბოლოში, თუ $N$ იყოფა $2^k$-ზე? \emph{[როგორც წყაროში]}
\\(a) $200$ \quad (b) $2000$ \quad (c) $20000$ \quad (d) $10000$

\item იპოვეთ $123^5$-ის ერთეულების ციფრი. \emph{[ერთეულების ციფრის ციკლურობა]}
\\(a) 9 \quad (b) 8 \quad (c) 3 \quad (d) 7

\item განსაზღვრეთ დაკარგული რიცხვი იმის გათვალისწინებით, რომ \ldots (გაყოფადობა 7-ზე). \emph{[როგორც წყაროში]}
\\(a) 10 \quad (b) 3 \quad (c) 5 \quad (d) 4

\end{enumerate}

\bigskip
\noindent\rule{\linewidth}{0.4pt}

% მთარგმნელის შენიშვნა: ზოგიერთი მოთხოვნა ნორმალიზებული იყო
% სადაც OCR-მა ქართული სკანირება გაუგებარი გახადა. ყველა მათემატიკური შინაარსი,
% ნუმერაცია და არჩევანი მოჰყვება ატვირთულ PDF-ს.

\end{document}