% !TEX TS-program = xelatex
\documentclass[12pt,oneside]{book}

\usepackage[a4paper,margin=1in]{geometry}

\usepackage{fontspec}
\defaultfontfeatures{Ligatures=TeX,Scale=MatchLowercase}
\usepackage{polyglossia}
\setdefaultlanguage{georgian}
\setotherlanguage{english}

% Georgian fonts - DejaVu Serif supports Georgian
\setmainfont{DejaVu Serif}
\newfontfamily\georgianfont{DejaVu Serif}[Script=Georgian]
\setsansfont{DejaVu Sans}
\setmonofont{DejaVu Sans Mono}

% Math: STIX Two Math
\usepackage{unicode-math}
\setmathfont{STIX Two Math}

\usepackage[protrusion=true,expansion=false]{microtype}
\usepackage{enumitem}
\setlist{nosep}
\usepackage{graphicx}
\usepackage[unicode]{hyperref}

\usepackage{fancyhdr}
\pagestyle{fancy}
\fancyhf{}
\fancyhead[LE,RO]{\thepage}
\fancyhead[LO,RE]{\nouppercase{\leftmark}}
\renewcommand{\headrulewidth}{0pt}
\renewcommand{\chaptermark}[1]{\markboth{#1}{}}
\renewcommand{\sectionmark}[1]{\markright{#1}}

% Georgian mathematical notations
\newcommand{\Z}{\mathbb{Z}}
\newcommand{\dotsep}{\,,\,}


\begin{document}

\section*{ნაწილი I}

\section{ნატურალური და მთელი რიცხვები}



ნატურალური რიცხვები, ჩვეულებრივ, საგანთა დასათვლელად გამოიყენება. უმცირესი ნატურალური რიცხვია 1. პირდადობის მიხედვით, ნატურალური რიცხვების ჩაწერა შეგვიძლია რიცხვთა
\begin{center}
1, 2, 3, 4, ...
\end{center}
მიმდევრობის სახით. ეს მიმდევრობა უსასრულოა, რადგან უდიდესი ნატურალური რიცხვი არ არსებობს.

ნატურალური რიცხვები, მათი მოპირდაპირე რიცხვები (ე.ი. -1, -2, ...) და ნული ერთობლიობაში წარმოადგენენ მთელ რიცხვებს. მთელი რიცხვების ჩაწერაც შეგვიძლია მიმდევრობის სახით:
\begin{center}
-3, -2, -1, 0, 1, 2, 3, ...
\end{center}

ეს მიმდევრობა უსასრულოდ გრძელდება ორივე მხარეს, რადგან არ არსებობს უდიდესი და უმცირესი მთელი რიცხვი.

\textbf{შეკრება.} ორი ან რამდენიმე რიცხვის შეკრების შედეგს ეწოდება მათი ჯამი, ხოლო თვით ამ რიცხვებს — მესაკრებები. მაგალითად, თუ $a+b+ \cdots +k=p$ მაშინ $p$ არის ჯამი, ხოლო $a, b, \ldots, k$ — შესაკრებები.

ნებისმიერად აღებული $a, b, c$ მთელი რიცხვებისათვის სრულდება ტოლობები
\begin{align}
a+b &= b+a, \quad \text{(გადანაცვლებადობის თვისება)},\\
(a+b)+c &= a+(b+c), \quad \text{(ჯუფდებადობის თვისება)}
\end{align}

ეს თვისებები ზოგჯერ მოსახერხებელია პრაქტიკული გამოთვლებისას.

\textbf{მაგალითი.} შევასრულოთ მოქმედებები: $1925+317+3075$
\begin{align}
&(1925+317)+3075 = (317+1925)+3075=317+(1925+3075) =\\
&=317+5000=5317.
\end{align}

\textbf{გამოკლება.} $a$ რიცხვს გამოვაკლოთ $b$ რიცხვი ნიშნავს მოვძებნოთ ისეთი $x$ რიცხვი, რომ $b+x=a$. ამ შემთხვევაში, $x$ რიცხვს ეწოდება $a$ და $b$ რიცხვების სხვაობა და აღინიშნება $a-–b$, $a$ რიცხვს ეწოდება საკლები, ხოლო $b$ ს — მაკლები.


მთელი რიცხვების სხვაობა ყოველთვის მთელი რიცხვია. ნატურალური რიცხვების სხვაობა შესაძლებელია არ იყოს ნატურალური რიცხვი. მაგალითად, $17-10$ ნატურალური რიცხვია, მაგრამ $17-17$, ისევე როგორც $17-20$, არ არის ნატურალური.

\textbf{გამრავლება.} ორი ან რამდენიმე რიცხვის ერთმანეთზე გამრავლების შედეგს ეწოდება მათი ნამრავლი, ხოლო თვით ამ რიცხვებს — მამრავლები (ან თანამამრავლები). ნებისმიერად აღებული $a, b, c$ მთელი რიცხვებისათვის სრულდება:
\begin{itemize}
\item გადანაცვლებადობის თვისება: $ab=ba$,
\item ჯუფდებადობის თვისება: $(ab)c=a(bc)$,
\item განრიგებადობის თვისება: $(a+b)c=ac+bc$
\end{itemize}

განრიგებადობის თვისების შედეგად გვაქვს
\[(a-b)c=ac-bc.\]

ამ თვისებებსაც პრაქტიკული ღირებულება აქვთ. მოვიყვანოთ მარტივი მაგალითი. შევასრულოთ მოქმედებები: $1756 \cdot 2179–1756 \cdot 2178$.
\[1756 \cdot 1=1756 \cdot (2179-2178)=1756 \cdot 2178-1756 \cdot 2179=1756 \cdot 1.\]

\textbf{გაყოფა.} $a$ რიცხვის გაყოფა არანულოვან $b$ რიცხვზე ნიშნავს ისეთი $x$ რიცხვის მოძებნას, რომ $b \cdot x=a$. ამ შემთხვევაში ვამბობთ, რომ $a$ რიცხვი იყოფა $b$ რიცხვზე. $a$-ს ეწოდება გასაყოფი, $a$-ს აგრეთვე მოვიხსენიებთ როგორც $b$-ს ჯერადს. $b$–ს ეწოდება გამყოფი. თვითონ $x$ რიცხვს ეწოდება $a$ და $b$ რიცხვების განაყოფი ანუ ფარდობა და აღინიშნება ასე: $a:b$, ან ასე: $\frac{a}{b}$.

ორი მთელი რიცხვის განაყოფი შესაძლოა არ აღმოჩნდეს მთელი რიცხვი. მაგალითად, $17:5$, ან $(−17):3$.

\textbf{ნატურალური ხარისხი.} განვსაზღვროთ $a^n$, ანუ $a$ რიცხვის $n$-ური ნატურალური ხარისხი. განმარტების თანახმად,
\[a^n = a \cdot a \cdot a \cdots a.\]

ანუ $a^n$ არის $n$ ცალი თანამამრავლის ნამრავლი, რომელთაგან თითოეული $a$-ს ტოლია. $a$-ს ეწოდება ხარისხის ფუძე, $n$-ს კი ხარისხის მაჩვენებელი. განმარტებიდან, კერძოდ, გამომდინარეობს $a^1=a$. თუ $a≠0$, მაშინ მიღებულია, რომ $a^0 =1$. შევნიშნოთ, რომ $0^0$ არ განიმარტება.

\textbf{1-ისა და 0-ის თვისებები.} თუ $a$ არის ნებისმიერი რიცხვი, მაშინ $a \cdot 0=0$, ნებისმიერ შემთხვევაში რაიმე გამოსახულების 1-ზე გამრავლება არ ცვლის მის მნიშვნელობას, ხოლო თუ $a$ არის არანულოვანი რიცხვი, მაშინ $a \cdot \frac{1}{a} = 1$. ამიტომ, 1-ის წარმოდგენა შეიძლება მრავალნაირად. მაგალითად $1=\frac{a}{a}$, ანუ $1=\frac{a}{a}$ ყოველი არანულოვანი $a$-სათვის.

რიცხვი 0 არც დადებითია და არც უარყოფითი. თუ $a$ ნებისმიერი რიცხვია, მაშინ $a+0=a$ და $a \cdot 0=0 \cdot a=0$. 0-ზე გაყოფა არ არის განსაზღვრული.

\section{ნატურალური რიცხვების წარმოდგენა თვლის ათობით სისტემაში}

თვლის ათობით სისტემაში ნატურალური რიცხვის ჩასაწერად გამოიყენება ათი ციფრი:
\begin{center}
0, 1, 2, 3, 4, 5, 6, 7, 8, 9.
\end{center}

თვლის ათობითი სისტემა არის პოზიციური სისტემა, რაც ნიშნავს, რომ ნატურალური რიცხვის ჩანაწერში მნიშვნელობა აქვს როგორც ციფრებს, ასევე იმ პოზიციას ანუ თანრიგს, რომელშიც დგას ციფრი. მაგალითად, 51 და 15 ერთიდაიგივე ციფრებისგან შედგება, მაგრამ განსხვავებული რიცხვებია, რადგან 51 არის 1 ერთეულისა და 5 ათეულის ჯამი, ხოლო 15 არის ხუთი ერთეულისა და 1 ათეულის ჯამი. ნებისმიერ შემთხვევაში, ნატურალური რიცხვის ჩანაწერში მარჯვენიდან პირველი თანრიგი არის ერთეულების, მისი მომდევნო ათეულების და ა.შ. მაგალითად 4193 ნიშნავს, რომ ეს რიცხვი არის 3 ერთეულის, 9 ათეულის, 1 ასეულისა და 4 ათასეულის ჯამი:
\[4193=3+9 \cdot 10+1 \cdot 100+4 \cdot 1000=3+9 \cdot 10^1+1 \cdot 10^2+4 \cdot 10^3.\]

ამავდროულად, უნდა შევნიშნოთ, რომ ეს რიცხვი შედგება 4193 ერთეულისაგან, 419 ათეულისაგან (რადგან $4193=3+419 \cdot 10$), 41 ასეულისაგან (რადგან $4193=93+41 \cdot 10^2$) და 4 ათასეულისაგან (რადგან $4193=193+4 \cdot 10^3$).

ზოგადად, ციფრის ჩასაწერად საკმარისია ერთი ასო, მაგალითად $a$, $b$ ან სხვა რომელიმე. ორნიშნა ნატურალურ რიცხვს წერენ $\overline{ab}$ სახით, $a$ და $b$ ციფრებია, $a≠0$ და
\[\overline{ab}=b+10a,\]

სამნიშნა რიცხვს ვწერთ სამი $a, b, c$ ციფრის საშუალებით, სადაც $a≠0$ და
\[\overline{abc}=c+10b+100a.\]

მეტი თანრიგების შემთხვევაშიც ანალოგიურად ვიქცევით. ასეთ ჩანაწერებში, მაგ $\overline{abcd}$, ზედა ხაზს ვიყენებთ, რადგან $abcd$ ჩანაწერი, როგორც წესი, ნიშნავს $a, b, c$ და $d$ ციფრების ნამრავლს.

\textbf{მაგალითი.} დავამტკიცოთ, რომ ნებისმიერი ორნიშნა რიცხვისა და იმ რიცხვის ჯამი, რომელიც შედგენილია იგივე ციფრებისაგან შებრუნებული მიმდევრობით, 11-ის ჯერადია.

\textbf{ამოხსნა.} ვთქვათ მოცემულია $\overline{ab}$.
\[\overline{ab}+\overline{ba}=b+10a+a+10b=11a+11b=11(a+b),\]

ანუ $\overline{ab}+\overline{ba}$ იყოფა 11-ზე.

\section{გაყოფადობის ნიშნები}

ხშირად, ერთი ნატურალური რიცხვის მეორე ნატურალურ რიცხვზე გაყოფის გარეშე შეგვიძლია დავადგინოთ იყოფა თუ არა პირველი რიცხვი მეორე რიცხვზე. ამაში გვეხმარება გაყოფადობის ნიშნები. მოვიყვანოთ ზოგიერთი მათგანი:

\textbf{1.} ვთქვათ $m$ არის რამდენიმე შესაკრების ჯამი, თუ თითოეული შესაკრები იყოფა $n$-ზე, მაშინ $m$-იც იყოფა $n$-ზე. მაგალითად, შეკრების გარეშე შეგვიძლია დავადგინოთ, რომ $48+64+96$ იყოფა 16-ზე, რადგან თითოეული შესაკრები იყოფა 16-ზე.

ამავე დროს, არ უნდა ჩავთვალოთ, რომ თუ რამდენიმე შესაკრები არ იყოფა რაღაც რიცხვზე, ჯამიც არ გაიყოფა ამ რიცხვზე. მაგალითად, $37+19$ ჯამი იყოფა 4-ზე, თუმცა არცერთი შესაკრები არაა 4-ის ჯერადი. ამრიგად, ჩამოყალიბებული ნიშანი გაყოფადობის საკმარისი, მაგრამ არა აუცილებელი, ნიშანია.

\textbf{2.} ვთქვათ $m$ არის რამდენიმე თანამამრავლის ნამრავლი. თუ ნამრავლის ერთ-ერთი თანამამრავლი მაინც იყოფა $n$-ზე, მაშინ ნამრავლიც იყოფა $n$-ზე. მაგალითად, გამრავლების გარეშე შეგვიძლია დავადგინოთ, რომ $105 \cdot 48 \cdot 93 \cdot 54$ იყოფა 5-ზე, რადგან 105 არის 5-ის ჯერადი.

ეს ნიშანიც საკმარისი ნიშანია მხოლოდ, მაგრამ არა აუცილებელი. მაგალითად, $12 \cdot 18$ იყოფა 36-ზე, მაგრამ არც 12 და არც 18 არ იყოფა 36-ზე.

\subsection{აუცილებელი და საკმარისი პირობები}

მათემატიკური მსჯელობა წარმოადგენს ერთიმეორისაგან გამომდინარე გამონათქვამების ერთობლიობას, თუ გვაქვს ორი $A$ და $B$ გამონათქვამი და $A$-დან გამომდინარეობს $B$, ამ ფაქტს სიმბოლურად ასე ჩაწერენ: $A \Rightarrow B$. თვით $\Rightarrow$ სიმბოლოს ლოგიკური გამომდინარეობის სიმბოლოს უწოდებენ.

თუ $A \Rightarrow B$, მაშინ $A$-ს ეწოდება $B$-ს საკმარისი პირობა, $B$-ს კი — $A$-ს აუცილებელი პირობა.

მაგალითად, პირობა „$a$ რიცხვი იყოფა 6-ზე\" არის საკმარისი პირობა იმისა, რომ $a$ რიცხვი გაიყოს 3-ზე, მაგრამ არ არის აუცილებელი. ამავე დროს პირობა „$a$ რიცხვი იყოფა 3-ზე" არის აუცილებელი პირობა იმისა, რომ $a$ რიცხვი გაიყოს 6-ზე, მაგრამ არ არის საკმარისი.

თუ $A \Rightarrow B$ და $B \Rightarrow A$ მაშინ ამბობენ, რომ $B$ არის $A$-ს აუცილებელი და საკმარისი პირობა (აგრეთვე, $A$ არის $B$-ს აუცილებელი და საკმარისი პირობა) ამ ფაქტს სიმბოლურად ასე ჩაწერენ: $A \Leftrightarrow B$, ამ სიმბოლოს კი ლოგიკური ტოლფასობის სიმბოლოს უწოდებენ.

\textbf{3.} 2-ზე იყოფა ის და მხოლოდ ის რიცხვი, რომელიც მთავრდება ლუწი ციფრით (0, 2, 4, 6, 8 ციფრებს ლუწი ციფრები ეწოდება, 1, 3, 5, 7, 9 ციფრებს კი — კენტი).

\textbf{4.} 3-ზე (9-ზე) იყოფა ის და მხოლოდ ის რიცხვი, რომლის ციფრთა ჯამი იყოფა 3-ზე (9-ზე).

\textbf{5.} 5-ზე იყოფა ის და მხოლოდ ის რიცხვი, რომელიც მთავრდება 0-ით ან 5-ით.

\textbf{6.} 10-ზე იყოფა ის და მხოლოდ ის რიცხვი, რომელიც მთავრდება 0-ით.

\section{კენტი და ლუწი რიცხვები}

ყოველი მთელი რიცხვი, რომელიც იყოფა 2-ზე, არის ლუწი რიცხვი. ლუწი მთელი რიცხვებია
\begin{center}
$\ldots, -4, -2, 0, 2, 4, 6, \ldots$
\end{center}

თუ მთელი რიცხვი არ იყოფა 2-ზე, მაშინ იგი არის კენტი რიცხვი. კენტი რიცხვების სიმრავლეა
\begin{center}
$\ldots, -3, -1, 1, 3, 5, \ldots$
\end{center}

თუ მთელი რიცხვების ნამრავლში ერთი თანამამრავლი მაინც არის ლუწი, მაშინ ნამრავლიც ლუწია, წინააღმდეგ შემთხვევაში ნამრავლი კენტია. თუ ორი მთელი რიცხვიდან ორივე ლუწია ან ორივე კენტია, მაშინ მათი ჯამი და სხვაობა ლუწია, წინააღმდეგ შემთხვევაში, მათი ჯამი და სხვაობა კენტია.

\textbf{მაგალითი.} პარლამენტში არის მხოლოდ ორი პარტია დეპუტატთა თანაბარი რაოდენობით. ერთ-ერთი კენჭისყრის შემდეგ, რომელშიც ყველა დეპუტატი მონაწილეობდა და თავი არავის შეუკავებია, გამოაცხადეს, რომ წინადადება მიღებულია 23 ხმის უპირატესობით. ამის შემდეგ ოპოზიციის ლიდერმა განაცხადა, რომ შედეგები გაყალბებულია. როგორ მიხვდა იგი ამას?

\textbf{ამოხსნა.} პირობის თანახმად, თითოეულ პარტიაში არის დეპუტატთა ტოლი რაოდენობა, ე.ი. დეპუტატების მთლიანი რაოდენობა ლუწია. მეორეს მხრივ, თუ წინააღმდეგი იყო $n$ დეპუტატი, მომხრე ყოფილა $n+23$ დეპუტატი, ხოლო დეპუტატთა მთლიანი რაოდენობა გამოდის
\[n+n+23=2n+23\]
ანუ კენტი რიცხვი, ე.ი. ვღებულობთ წინააღმდეგობას.

\section{მარტივი და შედგენილი რიცხვები}

მარტივი რიცხვი არის ისეთი ნატურალური რიცხვი, რომელსაც აქვს ზუსტად ორი გამყოფი, 1 და თვითონ ეს რიცხვი. მაგალითად, 2, 3, 5, 7, 11, 13 მარტივი რიცხვებია. 15 არაა მარტივი რიცხვი, რადგან აქვს ოთხი ნატურალური გამყოფი: 1, 3, 5, და 15. რიცხვი 1 არაა მარტივი, რადგან მხოლოდ ერთი ნატურალური გამყოფი აქვს. ნებისმიერი მთელი რიცხვი, რომელიც მეტია 1-ზე არის ან მარტივი, ან შედგენილი. შედგენილი რიცხვებისათვის სამართლიანია

\textbf{თეორემა.} ნებისმიერი შედგენილი ნატურალური რიცხვის წარმოდგენა შეიძლება მისი მარტივი გამყოფების ნამრავლის სახით და ეს წარმოდგენა ერთადერთია.

\begin{align}
20 &= 2 \cdot 2 \cdot 5,\\
14 &= 2 \cdot 7,\\
81 &= 3 \cdot 3 \cdot 3 \cdot 3,\\
484 &= 2 \cdot 2 \cdot 11 \cdot 11
\end{align}

შედგენილი რიცხვის მარტივ მამრავლებად გასაშლელად, მოსახერხებელია შემდეგი პროცედურის ჩატარება: რიცხვი იწერება მარცხნივ, მისი უმცირესი გამყოფი კი მარჯვნივ მის გასწვრივ, განაყოფი რიცხვის ქვეშ. განაყოფის გასწვრივ მისი უმცირესი მარტივი გამყოფი და ა.შ., როგორც ნაჩვენებია 525-თვის. შედეგად ვღებულობთ, რომ $525=3 \cdot 5 \cdot 5 \cdot 7=3 \cdot 5^2 \cdot 7$.

\begin{center}
\begin{tabular}{c|c}
525 & 3 \\
175 & 5 \\
35 & 5 \\
7 & 7 \\
1 & 
\end{tabular}
\end{center}

\section{უდიდესი საერთო გამყოფი}

რამდენიმე ნატურალური რიცხვის საერთო გამყოფი არის რიცხვი, რომელიც ყოველი მათგანის გამყოფს წარმოადგენს. $k, m, \ldots, n$ ნატურალური რიცხვების უდიდესი საერთო გამყოფი ეწოდება მათ საერთო გამყოფებს შორისს უდიდესს და იგი აღინიშნება $D(k, m,\ldots, n)$ სიმბოლოთი.

თუ $D(m,n)=1$, მაშინ $m$ და $n$-ს ეწოდებათ ურთიერთმარტივი.

შევნიშნოთ, რომ ორი მარტივი რიცხვი ყოველთვის ურთიერთმარტივია, თუმცა ორი შედგენილი რიცხვიც შეიძლება იყოს ურთიერთმარტივი, მაგალითად 8 და 9.

\textbf{მაგალითი.} იპოვეთ $D(126; 540; 630)$.

\textbf{ამოხსნა.} გავშალოთ მოცემული რიცხვები მარტივ მამრავლებად:

\begin{center}
\begin{tabular}{c|c c|c c|c}
126 & 2 & 540 & 2 & 630 & 2 \\
63 & 3 & 270 & 2 & 315 & 3 \\
21 & 3 & 135 & 3 & 105 & 3 \\
7 & 7 & 45 & 3 & 35 & 5 \\
1 &  & 15 & 3 & 7 & 7 \\
 &  & 5 & 5 & 1 &  \\
 &  & 1 &  &  & 
\end{tabular}
\end{center}

$126=2 \cdot 3^2 \cdot 7$, $540=2^2 \cdot 3^3 \cdot 5$, $630=2 \cdot 3^2 \cdot 5 \cdot 7$,

ამ გაშლებში რიცხვი 2 საერთო მამრავლად შედის ერთხელ, რიცხვი 3 ორჯერ, ხოლო 5 და 7 არ წარმოადგენენ საერთო მამრავლს, ამიტომ
\[D(126;540;630)=2 \cdot 3^2=18.\]

\section{უმცირესი საერთო ჯერადი}

რამდენიმე ნატურალური რიცხვის საერთო ჯერადი ეწოდება რიცხელს, რომელიც თითოეული მათგანის ჯერადს წარმოადგენს. $k, m, \ldots, n$ ნატურალური რიცხვების უმცირესი საერთო ჯერადი ეწოდება მათ საერთო ნატურალურ ჯერადთა შორის უმცირესს და იგი აღინიშნება $K(k,m,\ldots, n)$ სიმბოლოთი.

\textbf{მაგალითი.} იპოვეთ $K(270;300;315)$.

\textbf{ამოხსნა.} გავშალოთ მოცემული რიცხვები მარტივ მამრავლებად:

\begin{center}
\begin{tabular}{c|c c|c c|c}
270 & 2 & 300 & 2 & 315 & 3 \\
135 & 3 & 150 & 2 & 105 & 3 \\
45 & 3 & 75 & 3 & 35 & 5 \\
15 & 3 & 25 & 5 & 7 & 7 \\
5 & 5 & 5 & 5 & 1 &  \\
1 &  & 1 &  &  & 
\end{tabular}
\end{center}

$270=2 \cdot 3^3 \cdot 5$, $300=2^2 \cdot 3 \cdot 5^2$, $315=3^2 \cdot 5 \cdot 7$,

ამიტომ
\[K(270;300;315)=2^2 \cdot 3^3 \cdot 5^2 \cdot 7=18900.\]

\section{ნაშთი. ნაშთების არითმეტიკის ელემენტები}

თუ $m$ და $n$ არის ნატურალური რიცხვებია, მაშინ არსებობს ერთადერთი წყვილი მთელი რიცხვებისა $k$ და $r$, ისეთი რომ $n=m \cdot k+r$ და $0 \leq r < m$. $r$-ს ეწოდება ($n$-ის $m$-ზე გაყოფით მიღებული) ნაშთი. $k$-ს ეწოდება განაყოფი. მაგალითად, როდესაც 28-ს ვყოფთ 8-ზე, განაყოფი არის 3 და ნაშთი არის 4, რადგან $28=8 \cdot 3+4$. შევნიშნოთ, რომ $n$ არის $m$-ის ჯერადი მაშინ და მხოლოდ მაშინ, როდესაც ნაშთი არის 0-ის ტოლი. მაგალითად, 32-ის 8-ზე გაყოფის შედეგად ნაშთი მიიღება ნულის ტოლი, რადგან 32 არის 8-ის ჯერადი. შევნიშნოთ აგრეთვე, რომ როდესაც ერთ ნატურალურ რიცხვს ვყოფთ მასზე დიდ ნატურალურ რიცხვზე, განაყოფი არის ნული, ხოლო ნაშთი არის მცირე ნატურალური რიცხვი. მაგალითად, 5-ის 7-ზე გაყოფით მიიღება განაყოფი 0 და ნაშთი 5.

არსებობს რამდენიმე სასარგებლო წესი, რომელთა გამოყენებით შესაძლებელი ხდება ნაშთის დადგენა გაყოფის ან მთელი რიგი არითმეტიკული მოქმედებების შესრულების გარეშე.

\textbf{1)} ორი $a$ და $b$ რიცხვი მოცემულ $m$ რიცხვზე გაყოფისას მაშინ და მხოლოდ მაშინ იძლევა ერთიდაიგივე ნაშთს, როდესაც $a–b$ არის $m$-ის ჯერადი.

\textbf{2)} ჯამის რაიმე $m$ რიცხვზე გაყოფით მიღებული ნაშთი არ შეიცვლება, თუ ერთ შესაკრებს (ან თუნდაც ყველა შესაკრებს) შევცვლით სხვა რიცხვით, რომელიც $m$-ზე გაყოფისას იგივე ნაშთს იძლევა, რასაც ეს შესაკრები.

\textbf{მაგალითი.} მითითებული გამოთვლების შეუსრულებლად, იპოვეთ შემდეგი ჯამის 7-ზე გაყოფით მიღებული ნაშთი:
\[8+79+780+7781+77782+777783.\]

\textbf{ამოხსნა.} ამ შესაკრებების 7-ზე გაყოფით მიიღება ნაშთები: 1, 2, 3, 4, 5, და 6. მართლაც 1) წესის თანახმად,
$777783-6=777777$
არის 7-ის ჯერადი, და ასევე ვრწმუნდებით სხვა შესაკრებებისათვისაც. ახლა, 2) წესის თანახმად, ჩვენი მაგალითის ამოსახსნელად საკმარისია ვიპოვოთ
\[1+2+3+4+5+6=21\]
ჯამის 7-ზე გაყოფით მიღებული ნაშთი. რადგან 21 არის 7-ის ჯერადი, ამიტომ მაგალითში მოცემული ჯამიც არის 7-ის ჯერადი, ანუ უნაშთოდ იყოფა 7-ზე.

\textbf{3)} ნამრავლის რაიმე $m$ რიცხვზე გაყოფით მიღებული ნაშთი არ შეიცვლება, თუ ერთ თანამამრავლს (ან თუნდაც ყველა თანამამრავლს) შევცვლით სხვა რიცხვით, რომელიც $m$-ზე გაყოფისას იგივე ნაშთს იძლევა, რასაც ეს თანამამრავლი.

\textbf{მაგალითი.} რა იქნება შემდეგი ნამრავლის
\[7778 \cdot 7779 \cdot 7780 \cdot 7781 \cdot 7782 \cdot 7783\]
7-ზე გაყოფით მიღებული ნაშთი?

\textbf{ამოხსნა.} რადგან $7778–1=7777$ არის 7-ის ჯერადი, ამიტომ 7778 შეგვიძლია შევცვალოთ 1-ით, ანალოგიურად დანარჩენ თანამამრავლებს შევცვლით 2, 3, 4, 5, 6-ით
\[1 \cdot 2 \cdot 3 \cdot 4 \cdot 5 \cdot 6=720=102 \cdot 7+6\]
ე.ი. 720 7-ზე გაყოფისას იძლევა ნაშთს 6. მაშასადამე, მაგალითში მოცემული ნამრავლიც 7-ზე გაყოფისას იძლევა 6-ის ტოლ ნაშთს.

\textbf{მაგალითი.} რისი ტოლი იქნება $137^{100}$-ის ბოლო ციფრი?

\textbf{ამოხსნა.} ნებისმიერი ნატურალური რიცხვის ბოლო ციფრი არის ამ რიცხვის 10-ზე გაყოფით მიღებული ნაშთი. ამიტომ, რადგან $137$-იც და $7$-იც 10-ზე გაყოფისას იძლევიან ერთიდაიგივე ნაშთს, ამიტომ $137^{100}$ და $7^{100}$ რიცხვებიც 10-ზე გაყოფისას (3) წესის თანახმად) მოგვცემენ ერთიდაიგივე ნაშთს, ანუ $137^{100}$-ს და $7^{100}$-ს აქვთ ერთიდაიგივე ბოლო ციფრი.

$7^1$ მთავრდება 7-ით, $7^2$ მთავრდება 9-ით, $7^3$ მთავრდება 3-ით (რადგან $7 \cdot 9=63$), $7^4$ მთავრდება 1-ით, $7^5$ მთავრდება 7-ით და შემდეგ ეს მიმდევრობა (7, 9, 3, 1) პერიოდულად მეორდება. ამიტომ, 4, 8, 12 და ა.შ. 4-ის ჯერად ადგილებზე ამ მიმდევრობაში დგას 1-იანი. ამგვარად, მე-100 ადგილზეც დგას 1-იანი, ანუ $137^{100}$ მთავრდება 1-ით.

\section{თვლის ორობითი სისტემა. ათობითიდან ორობითში გადაყვანა და პირიქით}

როგორც ვიცით, ნატურალური რიცხვის ათობით ჩანაწერში ბოლო (მარჯვენა) ციფრი არის 10-ზე გაყოფისას მიღებული ნაშთი, ბოლო ორი ციფრი — 100-ზე გაყოფისას მიღებული ნაშთი და ა.შ., ანუ ათობითი ჩანაწერი მიიღება ამ რიცხვის 10-ზე გაყოფით, განაყოფის კვლავ 10-ზე გაყოფით და ა.შ. ვიდრე განაყოფში არ მივიღებთ ნულს და შემდეგ ნაშთების ერთად ამოწერით ვიღებთ ათობით ჩანაწერს.

ანალოგიურად, რიცხვის ორობითი ჩანაწერის მისაღებად (ანუ თვლის ორობით სისტემაში ჩასაწერად) ამ რიცხვს ვყოფთ 2-ზე, განაყოფს ისევ ვყოფთ 2-ზე და ა.შ., ვიდრე განაყოფში არ მივიღებთ ნულს. ნაშთებისაგან შედგენილი რიცხვი შეადგენს ამ რიცხვის ორობით ჩანაწერს. რადგან ნაშთი გამყოფზე ნაკლებია, ნებისმიერი ნატურალური რიცხვის ორობითი ჩანაწერი მხოლოდ 0-ების და 1-ებისაგან შეიძლება შედგებოდეს.

იმისათვის, რომ ორობითი და ათობითი ჩანაწერი არ აგვერიოს, ორობითის შემთხვევაში მივუთითებთ ფუძეს.

მაგალითად, ვნახოთ რას უდრის 100-ის ორობითი ჩანაწერი:

\begin{center}
\begin{tabular}{c|c}
100 & 2 \\
50 & 2 \\
25 & 2 \\
12 & 2 \\
6 & 2 \\
3 & 2 \\
1 & 2 \\
0 & 
\end{tabular}
\end{center}

ნაშთები: 0, 0, 1, 0, 0, 1, 1

ამიტომ $(1100100)_2$ არის 100 ორობით სისტემაში:
\[100 = (1100100)_2.\]

მაგალითად $(101)_2$, $(10010)_2$ არის ნატურალური რიცხვები, ხოლო $(10201)_2$, ან $(2031)_2$ უაზროა ჩანაწერებია.

ვნახოთ როგორ ხდება რიცხვის ორობითი ჩანაწერის ათობითში გადაყვანა, ან უფრო მარტივად რომ ვთქვათ, ორობითი ჩანაწერის ათობითი მნიშვნელობის გაგება.

ამ შემთხვევაშიც სასარგებლოა კარგად წარმოვიდგინოთ, თუ თვითონ ათობით სისტემაში ჩაწერილი რიცხვის მნიშვნელობას როგორ ვთვლით, მაგალითად:
\[1029 = 1 \cdot 10^3+0 \cdot 10^2+2 \cdot 10^1+9 \cdot 10^0 = 1000+20+9.\]

ანუ ჩანაწერის მნიშვნელობა ითვლება 10-ის ხარისხების გამოყენებით. ანალოგიურად ორობითი ჩანაწერის მნიშვნელობა ითვლება 2-ის ხარისხების გამოყენებით:
\[(10011)_2 = 1 \cdot 2^4+0 \cdot 2^3+0 \cdot 2^2+1 \cdot 2^1+1 \cdot 2^0 = 16+2+1=19.\]

\textbf{მაგალითი.} რომელია მეტი, 35 თუ $(110111)_2$?

\textbf{ამოხსნა.} რადგან $(110111)_2=1 \cdot 2^5+1 \cdot 2^4=32+16=48$, ხოლო $48>35$, ამიტომ $(110111)_2 > 35$.

\textbf{მაგალითი.} თუ თქვენმა მეგობარმა ჩაიფიქრა მთელი რიცხვი 1-სა და 1000-ს შორის, შეგიძლიათ თუ არა 10 კითხვაში (პასუხი შეიძლება იყოს „დიახ" და „არა") გამოიცნოთ იგი?

\textbf{ამოხსნა.} პირველი კითხვა: იყოფა თუ არა თქვენი რიცხვი 2-ზე უნაშთოდ? თუ კი, ვწერთ 0-ს, თუ არა 1-ს. მეორე კითხვა: გაყავით ორზე განაყოფი, რომელიც დაგვრჩა წინა კითხვაში 2-ზე გაყოფისას, რჩება ნაშთი? თუ არა ვწერთ 0-ს, თუ კი ვწერთ 1-ს. თუ ასე გავაგრძელებთ მივიღებთ ჩაფიქრებული რიცხვის ორობით ჩანაწერს.




\begin{center}
{\Large \textbf{ნაწილი II}}\\[0.25em]
\textbf{\S 1. ნატურალური და მთელი რიცხვები}
\end{center}

\section*{სავარჯიშოების ნაკრები A}

\begin{enumerate}[label=\textbf{\arabic*.\arabic*.}, leftmargin=2.2em, itemsep=0.35em]

% ---------- გვერდი 1 ----------
\item რომელი შემდეგი რიცხვებიდან არის მარტივი?
\\(1) 55\quad (2) 617\quad (3) 5027\quad (4) 42028

\item რომელი შემდეგი რიცხვებიდან არის შედგენილი?
\\(1) 389\quad (2) 1507\quad (3) 323061\quad (4) 52713

\item განსაზღვრეთ შემდეგი რიცხვების ლუწობა (ლუწი/კენტი):
\\(1) 531\quad (2) 2399\quad (3) 4057\quad (4) 20712

\item აირჩიეთ სწორი დებულებები მთელ რიცხვებზე:
\\(1) და (2) \emph{[დებულების ვარიანტები წყაროში]}\\
(3) \emph{შესაძლებელია ჩამოვაყალიბოთ \ldots}\quad
(4) \emph{ნული არც დადებითია და არც უარყოფითი.}

\item განსაზღვრეთ თითოეული შედეგის ნიშანი (დადებითი/უარყოფითი/ნული) ზუსტი მნიშვნელობის გამოთვლის \emph{გარეშე}.
\\(კონტექსტი: ჯამები/ნამრავლები მითითებული ნიშნებით.)

\item რომელი შემდეგთაგანია მთელი რიცხვები?
\\(1) $a+b$\quad (2) $a-b$\quad (3) $c+d$\quad (4) $\tfrac{c}{d}$\quad
(5) $cd$\quad (6) $a+c$\quad (7) $a-c$\quad (8) $ac$\quad (9) $2c+a$\quad
(10) $2c-3a$\quad (11) $a^2-c^2$\quad (12) $3a^2+2c^2$

\item გამოთვალეთ (ოპერაციების თანმიმდევრობა და ფრჩხილები):
\\(1) $24$\quad (2) $77$\quad (3) $128$\quad (4) $4128$ \hfill\emph{[მოკლე პრაქტიკის ამოცანები]}

\item იპოვეთ შემდეგი წყვილების (ან სიმრავლეების) უმცირესი საერთო ჯერადი (უსჯ).
\\(1) 28\quad (2) 99\quad (3) 200\quad (4) 532

\item იპოვეთ შემდეგი მთელი რიცხვების უდიდესი საერთო გამყოფი (უსგ).
\\(1) 12\quad (2) 13\quad (3) 50\quad (4) 120

\item რამდენი ნატურალური გამყოფი აქვს $66$-ს?

\item იპოვეთ მოცემული რიცხვების საერთო გამყოფების რაოდენობა.
\\(1) 15\quad (2) 19\quad (3) 27\quad (4) 48

\item გამოთვალეთ თითოეული წყვილის უსჯ:
\\(1) $12$ და $42$\quad (2) $240$ და $360$\quad (3) $1680$ და $4200$\quad (4) $180,\ 270,\ 450$

\item გამოთვალეთ თითოეული ჯგუფის უსგ:
\\(1) $12$ და $15$\quad (2) $24,\ 30,\ 60$\quad (3) $7,\ 8,\ 13$\quad (4) $20,\ 65,\ 125$

\item გაამარტივეთ შემდეგი გამოსახულებები:
\\(1) $2^5\cdot 3^7\cdot 5^{-3}$ და $2^1\cdot 3^3\cdot 5^2$
\\(2) $2^{-7}\cdot 3^2$ და $2^2\cdot 7^3$ \hfill\emph{[სტანდარტიზებული ხარისხები წყაროს მიხედვით]}

% ---------- გვერდი 2 ----------
\setcounter{enumi}{15}

\item სიტყვიერი ამოცანები გაყოფადობაზე (პარაფრაზირებული სკანირებიდან):
\\(1) თუ რიცხვი ტოვებს ნაშთს $n$ გაყოფისას $m$-ზე, რა არის \ldots ნაშთი?
\\(2) თუ მთელი რიცხვი იზრდება $+2$-ით, როგორ იცვლება ლუწობა?
\\(3) \emph{[გაყოფადობა ჯამებით]} \quad
(4) თუ მთელი რიცხვი იზრდება $+1$-ით, როგორ \ldots?

\item ააგეთ უმცირესი (ან უდიდესი) $n$-ნიშნა რიცხვი, რომელიც აკმაყოფილებს მოცემულ ციფრულ შეზღუდვებს. \emph{[როგორც წყაროში]}

\item რამდენი მთელი რიცხვი აკმაყოფილებს მოცემულ უტოლობას/ინტერვალს? \emph{[მთელი ამონახსნების რაოდენობა]}

\item კლასის ამოცანა დაჯდომა/განლაგების შესახებ ჯამებით 30, 40 და 47. განსაზღვრეთ რამდენი \ldots \emph{[როგორც წყაროში]}

\item არითმეტიკა შერეული ოპერაციებით (აირჩიეთ სწორი შედეგი):
\\(1) $1*00$\quad (2) $25*8$\quad (3) $4*1$\quad (4) $*888$
\\\emph{[აქ $*$ აღნიშნავს თანდასმას წყაროში; გადააკეთეთ საჭიროების მიხედვით კლასში.]}

\item განსაზღვრეთ ხაზგასმული ციფრის ადგილობრივი მნიშვნელობა თითოეულ რიცხვში:
\\(1) $\underline{54312}$ \quad (2) $4532\underline{\phantom{0}}$ \quad (3) $32*25$ \quad
(4) $*3260$ \quad (5) $423*0$ \quad (6) $*2310$ \emph{[პოზიციები წყაროს მიხედვით]}

\item ჩაწერეთ შემდეგი დიდი რიცხვები დაჯგუფებით (ინტერვალებით) და სიტყვებით:
\\(1) 1275\quad (2) 33333\quad (3) 10203040\quad (4) 1919191919

\item ორი სიტყვიერი ამოცანა, რომელიც მოიცავს დრო/სიჩქარის სცენარებს „5 დღე" და „7 დღე" (უსჯ/უსგ გამოყენებები). \emph{[როგორც წყაროში]}

\item თუ $a$ და $b$ არის მთელი რიცხვები და $a-b$ იყოფა $5$-ზე, განსაზღვრეთ, რომელი გამოსახულებები იყოფა $5$-ზე. \emph{[წყაროს მიხედვით]}

\item თუ მთელი რიცხვი იყოფა $17$-ზე და სხვა დაკავშირებული რაოდენობა უდრის $8$-ს, რას შეიძლება ითქვას $a+b$ შესახებ მოდულო $17$? \emph{[წყაროს მიხედვით]}

\item მოკლე პასუხები:
\\(1) \emph{აირჩიეთ სწორი ვარიანტი ნიშნის/ზომის საფუძველზე.}
\\(2) დამრგვალების/შეფასების ამოცანა უახლოეს 10-მდე; შემდეგ უახლოეს 100-მდე. \emph{[როგორც წყაროში]}

\item დაალაგეთ შემდეგი მთელი რიცხვები რიცხვით ხაზზე:
\\(1) $-2$-დან $0$-მდე \quad (2) $-2$-დან $3$-მდე \quad (3) $-5$-დან $1$-მდე \quad (4) $-1$-დან $9$-მდე

\item გამოთვალეთ:
\\(1) $-2+5-6$\quad (2) $-11-(-3)$\quad (3) $14-(-4)$\quad (4) $-17+(+2)$\\
(5) $(-2-3)+2+5$\quad (6) $(-2+6)+(-6)$\quad (7) $(-2-5)-(+3-7)$\quad (8) $(-10+3-4)(4-6)$

\item გამოთვალეთ:
\\(1) $3(9-11)-4(8-9)$ \quad (2) $-5(-7-5)+3(6-9)$ \quad (3) $\dfrac{5\,(9-13)\,(-4)}{-8\,(9-14)}$ \quad (4) $\dfrac{-7\,(20-11)\cdot 2}{3\,(2-16)}$

% ---------- გვერდი 3 ----------
\setcounter{enumi}{30}

\item გამოთვალეთ:
\\(1) $36:3-9$ \quad (2) $36:(3-9)$ \quad (3) $30-15:3$ \quad (4) $(30-15):3$\\
(5) $240-25:(13-8)$ \quad (6) $120:(10-16)-20$ \quad (7) $50-(-20):5$ \quad (8) $(50-(-20)):5$

\item სიტყვიერი ამოცანები:
\\(1) მუშაკი ხარჯავს 6 დღეს ერთ ამოცანაზე და 70 გვერდს მეორეზე; თუ 50 გვერდი არის \ldots განსაზღვრეთ რამდენი დღე სჭირდება \ldots
\\(2) რიცხვი $152\;2\underline{\phantom{0}}3$ იყოფა 16-ზე; იპოვეთ დაკარგული ციფრი. \emph{[როგორც წყაროში]}

\item კლასში A არის 10-ით მეტი მოსწავლე, ვიდრე კლასში B. თუ \ldots როდესაც კლასები გაერთიანდება \ldots იპოვეთ A-ის და B-ის ზომები. \emph{[როგორც წყაროში]}

\item მართკუთხედი პერიმეტრის ინფორმაციით; იპოვეთ გვერდების სიგრძეები მოცემული კავშირებით და იმით, რომ პერიმეტრი არის 28. \emph{[როგორც წყაროში]}

\item მატარებელი გადის 59 კმ პირველ საათში და \ldots განსაზღვრეთ საერთო მანძილი 10 საათის შემდეგ. \emph{[როგორც წყაროში]}

\item დავუშვათ $x,y,z$ არის მთელი რიცხვები. თუ $xy-3z$ იყოფა \ldots განსაზღვრეთ არის თუ არა \ldots

\item აირჩიეთ სწორი სამეცნიერო ნოტაცია (ან სიდიდე) თითოეული რიცხვისთვის:
\\(1) $10$ \quad (2) $10\,000$ \quad (3) $41$

\item გამოთვალეთ კვადრატები (შაბლონის ამოცნობა):
\\(1) $(11)^2$ \quad (2) $(111)^2$ \quad (3) $(1000)^2$ \quad (4) $(10000)^2$

\end{enumerate}

\bigskip
\section*{საკონტროლო ტესტი N1 (A)}

\begin{enumerate}[label=\textbf{\arabic*.}, leftmargin=2.2em, itemsep=0.35em, resume*]

\item რომელი შემდეგი რიცხვებიდან იყოფა $6$-ზე?
\\(a) 273 \quad (b) 412 \quad (c) 1002 \quad (d) 3004

\item გამოთვალეთ: $400:(25-75)-200:(13-33)$
\\(a) $2$ \quad (b) $18$ \quad (c) $-18$ \quad (d) $-2$

\item რომელ წელს 2011-სა და \ldots შორის ხდება \ldots რაოდენობა გაყოფადი $5$-ზე?
\\(a) 2 \quad (b) 5 \quad (c) 3

\item დავუშვათ $a,b,c,d$ განისაზღვრება $a=3\cdot(5-7)$, $b=3-5-7$, $c=ab$, $d=a+b$. აირჩიეთ სწორი თანმიმდევრობა:
\\(a) $a,c,b,d$ \quad (b) $c,a,d,b$ \quad (c) $c,d,b,a$ \quad (d) $c,d,a,b$

\item რიცხვი $123$ იზრდება 8-ით და შემდეგ მცირდება 3-ით. რომელ ინტერვალში მოექცევა შედეგი?
\\(a) 10 \quad (b) 15 \quad (c) 20 \quad (d) 12

\item სკოლაში არის 155 პირველკლასელი და 62 \ldots რომელია უახლოესი მარტივი რიცხვი?
\\(a) 20 \quad (b) 25 \quad (c) 27 \quad (d) 31

\item რომელი რიცხვია $15$-ის ჯერადი?
\\(a) 5005 \quad (b) 4005 \quad (c) 1015 \quad (d) 2015

\item იპოვეთ $101$-ის უახლოესი მარტივი რიცხვი მარჯვნიდან და მარცხნიდან.
\\(a) 909 \quad (b) 816 \quad (c) 807 \quad (d) 809

\item აირჩიეთ სწორი ფაქტორიზაცია:
\\(a) $5$-რაღაც \quad (b) \emph{[ვარიანტები წყაროს მიხედვით: ორწევრიანი vs სამწევრიანი ფაქტორიზაცია]}

\item რა არის $36$-ის უმცირესი გამყოფი?
\\(a) 1 \quad (b) 2 \quad (c) 3 \quad (d) 4

\item წრფივი გამოსახულება ფორმით $2n+3$. რომელი შემდეგთაგანია შესაძლო მნიშვნელობები?
\\(a) $6n+1$ \quad (b) $6n+3$ \quad (c) $3n+3$ \quad (d) $6n-1$

\item რამდენი ნული არის $N$-ის ბოლოში, თუ $N$ იყოფა $2^k$-ზე? \emph{[როგორც წყაროში]}
\\(a) $200$ \quad (b) $2000$ \quad (c) $20000$ \quad (d) $10000$

\item იპოვეთ $123^5$-ის ერთეულების ციფრი. \emph{[ერთეულების ციფრის ციკლურობა]}
\\(a) 9 \quad (b) 8 \quad (c) 3 \quad (d) 7

\item განსაზღვრეთ დაკარგული რიცხვი იმის გათვალისწინებით, რომ \ldots (გაყოფადობა 7-ზე). \emph{[როგორც წყაროში]}
\\(a) 10 \quad (b) 3 \quad (c) 5 \quad (d) 4

\end{enumerate}

\bigskip
\noindent\rule{\linewidth}{0.4pt}

% მთარგმნელის შენიშვნა: ზოგიერთი მოთხოვნა ნორმალიზებული იყო
% სადაც OCR-მა ქართული სკანირება გაუგებარი გახადა. ყველა მათემატიკური შინაარსი,
% ნუმერაცია და არჩევანი მოჰყვება ატვირთულ PDF-ს.

\end{document}